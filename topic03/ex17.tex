\documentclass[titlepage, letterpaper, fleqn]{article}
\usepackage[utf8]{inputenc}
\usepackage{fancyhdr} % fancy headers, of course!
\usepackage{amsmath} % what do you think?
\usepackage{amsthm} % theorems!
\usepackage{extramarks} % more cute things
\usepackage{enumitem} % i'm not sure...
\usepackage{multicol} % multicolumn...?
\usepackage{amssymb} % more symbols
\usepackage{MnSymbol} % moar symbols?
\usepackage{booktabs} % cool looking tables
\usepackage{tikz} %venn and shizzle
\usepackage{tikz-qtree-compat} %tableaux
\usepackage{lipsum} %lorem ipsum dolor sit amet f u
\usepackage{mathrsfs} %math script for calligraphic scripting, I GUESS

\topmargin=-0.45in
\evensidemargin=0in
\oddsidemargin=0in
\textwidth=6.5in
\textheight=9.0in
\headsep=0.25in


%
% You should change this things~
%

\newcommand{\mahteacher}{Dr. Viacheslav Kalashnikov}
\newcommand{\mahclass}{Applied Mathematics}
\newcommand{\mahtitle}{Topic III - Activity 17}
\newcommand{\mahdate}{November 09, 2016}
\newcommand{\spacepls}{\vspace{5mm}}
\newcommand{\until}{\mathscr{U}}
\renewcommand\qedsymbol{\(\blacksquare\)}

%
% Header markings
%

\pagestyle{fancy}
\lhead{1170065 - Xavier Sánchez}
\chead{}
\rhead{}
\lfoot{}
\rfoot{}


\renewcommand\headrulewidth{0.4pt}
\renewcommand\footrulewidth{0.4pt}

\setlength\parindent{0pt}


%
% Create Problem Sections (stolen directly from jdavis/latex-homework-template @ github!)
%

\newcommand{\enterProblemHeader}[1]{
\nobreak\extramarks{}{Problem \arabic{#1} continued on next page\ldots}\nobreak{}
\nobreak\extramarks{Problem \arabic{#1} (continued)}{Problem \arabic{#1} continued on next page\ldots}\nobreak{}
}

\newcommand{\exitProblemHeader}[1]{
\nobreak\extramarks{Problem \arabic{#1} (continued)}{Problem \arabic{#1} continued on next page\ldots}\nobreak{}
\stepcounter{#1}
\nobreak\extramarks{Problem \arabic{#1}}{}\nobreak{}
}

\setcounter{secnumdepth}{0}
\newcounter{partCounter}
\newcounter{homeworkProblemCounter}
\setcounter{homeworkProblemCounter}{1}
\nobreak\extramarks{Exercise \arabic{homeworkProblemCounter}}{}\nobreak{}

%Solution Environment
\newenvironment{solution}
{\renewcommand\qedsymbol{$\square$}\begin{proof}[Solution]}
{\end{proof}}

% Alias for the Solution section header
%\newcommand{\solution}{\textbf{\Large Solution}}

%Alias for the new step section
\newcommand{\steppy}[1]{\textbf{\large #1}}

%
% Homework Problem Environment
%
% This environment takes an optional argument. When given, it will adjust the
% problem counter. This is useful for when the problems given for your
% assignment aren't sequential. See the last 3 problems of this template for an
% example.
%
\newenvironment{homeworkProblem}[1][-1]{
\ifnum#1>0
\setcounter{homeworkProblemCounter}{#1}
\fi
\section{Exercise \arabic{homeworkProblemCounter}}
\setcounter{partCounter}{1}
\enterProblemHeader{homeworkProblemCounter}
}{
\exitProblemHeader{homeworkProblemCounter}
}

%
% My actual info
%

\title{
\vspace{1in}
\textbf{Tecnológico de Monterrey} \\
\vspace{0.5in}
\textmd{\mahclass} \\
\large{\textit{\mahteacher}} \\
\vspace{0.5in}
\textsc{\mahtitle}\\
\textsc{3.7.1 Probability distributions}\\
\textsc{3.7.2 Probability distributions}\\
\textsc{3.7.3 Probability distributions}\\
\author{01170065  - MIT \\
Xavier Fernando Cuauhtémoc Sánchez Díaz \\
\texttt{mail@gmail.com}}
\date{\mahdate}
}

\begin{document}

\begin{titlepage}
\maketitle
\end{titlepage}

%
% Actual document starts here~
%

\section{Exercise 3.7.1}

{\large Suppose someone who knows 60\% of the material covered in a chapter of a textbook is taking a five question objective (each answer is either right or wrong) quiz.
Let \(X\) be the random variable that for each possible quiz,
gives the number of questions the student answers correctly.

What is the expected value of the random variable \(X - 3\)?
What is the expected value of \((X - 3)^2\)?
What is the variance of \(X\)?}

\begin{solution}
This is an example of a binomial distribution, and therefore the expected value \(E(X - 3) = E(X) - 3 = 0\), since \(E(X) = 0.6 \cdot 5 = 3\).

\begin{align*}
E(X -3)^2 & = \binom{5}{1} * 0.6 * 0.4 \\
& = 1.2
\end{align*}

The variance is equal to the value of \(E(X-3)^2\), 1.2.
\end{solution}

\spacepls

\section{Exercise 3.7.2}

{\large In the last exercise, let \(X_i\) be the number of correct answers the student gets on question \(i\), so that \(X_i\) is either zero or one.
What is the expected value of \(X_i\)?
What is the variance of \(X_i\)?
How does the sum of the variances of \(X_1\) through \(X_5\) relate to the variance of \(X\) on the last exercise?}

\begin{solution}
The expected value of \(X_i\) is calculated as \(E(X_i) = 1 * 0.6 = 0.6\).

Now the variance of \(X_i\) is calculated as 2 choose 2 (correct or incorrect answer) multiplied by the probability of a correct answer and multiplied by the probability of an incorrect answer:
\(\binom{2}{2} * 0.6 * 0.4 = 0.24\).

The sum of variances from \(X_1, \dots , X_5\) is equal to 1.2, which is the same on the last exercise.
\end{solution}

\spacepls

\section{Exercise 3.7.3}

{\large We have a dime and a fifty cent piece in a cup.
We withdraw one coin.

\textbf{a)} What is the expected amount of money we withdraw? What is the variance?}

\begin{solution}
The expected amount is \(E(X) = 0.5 * 10 + 0.5 * 50 = 30\) cents.

The variance is \(\sigma^2 = 2 * 0.5 * 0.5 = 0.5\).
\end{solution}

\spacepls

{\large \textbf{b)} Now we draw a second coin, without replacing the first.
What is the expected amount of money we withdraw? What is the variance?}

\begin{solution}
Both the expected amount and the variance are the same, \(E(X) = 0.5 * 10 + 0.5 * 50 = 30\) cents.

The variance is \(\sigma^2 = 2 * 0.5 * 0.5 = 0.5\).
\end{solution}

\pagebreak

{\large \textbf{c)} Suppose instead we consider withdrawing two coins from the cup together. What is the expected amount of money we withdraw, and what is the variance?}

\begin{solution}
The expected amount is \(E(2X) = 2E(X) = 2 * 0.5 * 10 + 0.5 * 50 = 60\) cents, 30 per coin.
Getting both coins would double the expected value, and in this case gives us the maximum amount possible, 60.
The variance is \(\sigma^2 = 2 * 0.5 * 0.5 = 0.5\) also doubled, so it becomes 1.
\end{solution}
\end{document}