\documentclass[titlepage, letterpaper, fleqn]{article}
\usepackage[utf8]{inputenc}
\usepackage{fancyhdr} % fancy headers, of course!
\usepackage{amsmath} % what do you think?
\usepackage{amsthm} % theorems!
\usepackage{extramarks} % more cute things
\usepackage{enumitem} % i'm not sure...
\usepackage{multicol} % multicolumn...?
\usepackage{amssymb} % more symbols
\usepackage{MnSymbol} % moar symbols?
\usepackage{booktabs} % cool looking tables
\usepackage{tikz} %venn and shizzle
\usepackage{tikz-qtree-compat} %tableaux
\usepackage{lipsum} %lorem ipsum dolor sit amet f u
\usepackage{mathrsfs} %math script for calligraphic scripting, I GUESS

\topmargin=-0.45in
\evensidemargin=0in
\oddsidemargin=0in
\textwidth=6.5in
\textheight=9.0in
\headsep=0.25in


%
% You should change this things~
%

\newcommand{\mahteacher}{Dr. Viacheslav Kalashnikov}
\newcommand{\mahclass}{Applied Mathematics}
\newcommand{\mahtitle}{Topic III - Activity 11}
\newcommand{\mahdate}{October 26, 2016}
\newcommand{\spacepls}{\vspace{5mm}}
\newcommand{\until}{\mathscr{U}}
\renewcommand\qedsymbol{\(\blacksquare\)}

%
% Header markings
%

\pagestyle{fancy}
\lhead{1170065 - Xavier Sánchez}
\chead{}
\rhead{}
\lfoot{}
\rfoot{}


\renewcommand\headrulewidth{0.4pt}
\renewcommand\footrulewidth{0.4pt}

\setlength\parindent{0pt}


%
% Create Problem Sections (stolen directly from jdavis/latex-homework-template @ github!)
%

\newcommand{\enterProblemHeader}[1]{
\nobreak\extramarks{}{Problem \arabic{#1} continued on next page\ldots}\nobreak{}
\nobreak\extramarks{Problem \arabic{#1} (continued)}{Problem \arabic{#1} continued on next page\ldots}\nobreak{}
}

\newcommand{\exitProblemHeader}[1]{
\nobreak\extramarks{Problem \arabic{#1} (continued)}{Problem \arabic{#1} continued on next page\ldots}\nobreak{}
\stepcounter{#1}
\nobreak\extramarks{Problem \arabic{#1}}{}\nobreak{}
}

\setcounter{secnumdepth}{0}
\newcounter{partCounter}
\newcounter{homeworkProblemCounter}
\setcounter{homeworkProblemCounter}{1}
\nobreak\extramarks{Exercise \arabic{homeworkProblemCounter}}{}\nobreak{}

% Alias for the Solution section header
\newcommand{\solution}{\textbf{\Large Solution}}

%Alias for the new step section
\newcommand{\steppy}[1]{\textbf{\large #1}}

%
% Homework Problem Environment
%
% This environment takes an optional argument. When given, it will adjust the
% problem counter. This is useful for when the problems given for your
% assignment aren't sequential. See the last 3 problems of this template for an
% example.
%
\newenvironment{homeworkProblem}[1][-1]{
\ifnum#1>0
\setcounter{homeworkProblemCounter}{#1}
\fi
\section{Exercise \arabic{homeworkProblemCounter}}
\setcounter{partCounter}{1}
\enterProblemHeader{homeworkProblemCounter}
}{
\exitProblemHeader{homeworkProblemCounter}
}

%
% My actual info
%

\title{
\vspace{1in}
\textbf{Tecnológico de Monterrey} \\
\vspace{0.5in}
\textmd{\mahclass} \\
\large{\textit{\mahteacher}} \\
\vspace{0.5in}
\textsc{\mahtitle}\\
\textsc{3.1.1 Counting and Probability}\\
\textsc{3.1.2 Counting and Probability}\\
\author{01170065  - MIT \\
Xavier Fernando Cuauhtémoc Sánchez Díaz \\
\texttt{xavier.sanchezdz@gmail.com}}
\date{\mahdate}
}

\begin{document}

\begin{titlepage}
\maketitle
\end{titlepage}

%
% Actual document starts here~
%

\section{Exercise 3.1.1}

{\large \textbf{a)} How many positive two-digit integers are multiples of 3?}

\spacepls

aa

{\large \textbf{b)} What is the probability that a randomly chosen positive two-digit integer is a multiple of 3?}

\spacepls

aa

{\large \textbf{c)} What is the probability that a randomly chosen positive two-digit integer is a multiple of 4?}

\spacepls

aa

{\large \textbf{d)} How many positive three-digit integers are multiples of 6?}

\spacepls

aa

{\large \textbf{e)} What is the probability that a randomly chosen positive three-digit integer is a multiple of 7?}

\spacepls

aa

\section{Exercise 3.1.2}

{\large Suppose \(A[1], A[2], A[3], \dots, A[n]\) is a one-dimensional array and \(n \geq 50\).\\
\textbf{a)} How many elements are in the array?}

\spacepls

aa

{\large \textbf{b)} How many elements are in the subarray \(A[4], A[5],\dots , A[39]\)?}

\spacepls

aa

{\large \textbf{c)} If \(3 \leq m \leq n\), what is the probability that a randomly chosen array element is in the subarray \(A[3], A[4], \dots, A[m]\)}?

\spacepls

aa

{\large \textbf{d)} What is the probability that a randomly chosen array element is in the subarray shown below if \(n = 39\)?\\
\(A[\frac{n}{2}], A[\frac{n}{2} + 1], \dots , A[n]\)}

\spacepls

aa

\end{document}