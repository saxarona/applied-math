\documentclass[titlepage, letterpaper, fleqn]{article}
\usepackage[utf8]{inputenc}
\usepackage{fancyhdr} % fancy headers, of course!
\usepackage{amsmath} % what do you think?
\usepackage{amsthm} % theorems!
\usepackage{extramarks} % more cute things
\usepackage{enumitem} % i'm not sure...
\usepackage{multicol} % multicolumn...?
\usepackage{amssymb} % more symbols
\usepackage{booktabs} % cool looking tables
\usepackage{tikz} %venn and shizzle

\topmargin=-0.45in
\evensidemargin=0in
\oddsidemargin=0in
\textwidth=6.5in
\textheight=9.0in
\headsep=0.25in


%
% You should change this things~
%

\newcommand{\mahteacher}{Dr. Viacheslav Kalashnikov}
\newcommand{\mahclass}{Applied Mathematics}
\newcommand{\mahtitle}{Activities 9, 10, 11 \& 12}
\newcommand{\mahdate}{August 31, 2016}
\newcommand{\spacepls}{\vspace{5mm}}
\renewcommand\qedsymbol{\(\blacksquare\)}

%
% Header markings
%

\pagestyle{fancy}
\lhead{1170065 - Xavier Sánchez}
\chead{}
\rhead{}
\lfoot{}
\rfoot{}


\renewcommand\headrulewidth{0.4pt}
\renewcommand\footrulewidth{0.4pt}

\setlength\parindent{0pt}


%
% Create Problem Sections (stolen directly from jdavis/latex-homework-template @ github!)
%

\newcommand{\enterProblemHeader}[1]{
\nobreak\extramarks{}{Problem \arabic{#1} continued on next page\ldots}\nobreak{}
\nobreak\extramarks{Problem \arabic{#1} (continued)}{Problem \arabic{#1} continued on next page\ldots}\nobreak{}
}

\newcommand{\exitProblemHeader}[1]{
\nobreak\extramarks{Problem \arabic{#1} (continued)}{Problem \arabic{#1} continued on next page\ldots}\nobreak{}
\stepcounter{#1}
\nobreak\extramarks{Problem \arabic{#1}}{}\nobreak{}
}

\setcounter{secnumdepth}{0}
\newcounter{partCounter}
\newcounter{homeworkProblemCounter}
\setcounter{homeworkProblemCounter}{1}
\nobreak\extramarks{Exercise \arabic{homeworkProblemCounter}}{}\nobreak{}

% Alias for the Solution section header
\newcommand{\solution}{\textbf{\Large Solution}}

%Alias for the new step section
\newcommand{\steppy}[1]{\textbf{\large #1}}

%
% Homework Problem Environment
%
% This environment takes an optional argument. When given, it will adjust the
% problem counter. This is useful for when the problems given for your
% assignment aren't sequential. See the last 3 problems of this template for an
% example.
%
\newenvironment{homeworkProblem}[1][-1]{
\ifnum#1>0
\setcounter{homeworkProblemCounter}{#1}
\fi
\section{Exercise \arabic{homeworkProblemCounter}}
\setcounter{partCounter}{1}
\enterProblemHeader{homeworkProblemCounter}
}{
\exitProblemHeader{homeworkProblemCounter}
}

%
% Venn diagrams defs
%

% \def\firstcircle{(0,0) circle (1.5cm)}
% \def\secondcircle{(0:2cm) circle (1.5cm)}
% \colorlet{circle edge}{blue!50}
% \colorlet{circle area}{blue!20}

% \tikzset{filled/.style={fill=circle area, draw=circle edge, thick},
%     outline/.style={draw=circle edge, thick}}

%
% My actual info
%

\title{
\vspace{1in}
\textbf{Tecnológico de Monterrey} \\
\vspace{0.5in}
\textmd{\mahclass} \\
\large{\textit{\mahteacher}} \\
\vspace{0.5in}
\textsc{\mahtitle}\\
\textsc{1.3.1: Functions: Image, Closure}\\
\textsc{1.3.2: Injections, surjections, bijections}\\
\textsc{1.3.3: Pigeonhole principle}\\
\textsc{1.3.4: Handy functions}\\
\author{01170065  - MIT \\
Xavier Fernando Cuauhtémoc Sánchez Díaz \\
\texttt{xavier.sanchezdz@gmail.com}}
\date{\mahdate}
}

\begin{document}

\begin{titlepage}
\maketitle
\end{titlepage}

%
% Actual document starts here~
%

\section{Exercise 1.3.1}

{\large \textbf{a)} The \textit{floor} function from \(\mathbb{R}\) into \(\mathbb{N}\) is defined by putting \(\lfloor x \rfloor\) to be the largest integer less than or equal to \(x\).
What are the images under the floor function of the following sets?}\\
\([0,1] = \{x \in \mathbb{R} \colon 0 \leq x \leq 1\}\), \([0,1) = \{x \in \mathbb{R} \colon 0 \leq x < 1\}\),\\
\((0,1] = \{x \in \mathbb{R} \colon 0 < x \leq 1\}\), \((0,1) = \{x \in \mathbb{R} \colon 0 < x < 1\}\)

\[[0,1] = \{x \in \mathbb{R} \colon 0 \leq x \leq 1\}\]
\[\text{Image} = \{0,1\}\]

\[[0,1) = \{x \in \mathbb{R} \colon 0 \leq x < 1\}\]
\[\text{Image} = \{0\}\]

\[(0,1] = \{x \in \mathbb{R} \colon 0 < x \leq 1\}\]
\[\text{Image} = \{0,1\}\]

\[(0,1) = \{x \in \mathbb{R} \colon 0 < x < 1\}\]
\[\text{Image} = \{0\}\]

\spacepls

{\large \textbf{b)} Let \(f\colon A \to A\) be a function from set \(A\) to itself. Show that for all \(X \subseteq A, f(A) \subseteq f[A]\), and give a simple example of the failure of the converse inclusion.}

\begin{proof}
Let \(f\colon A \to A\) and \(X \subseteq A\).\\
By definition, closure \(f[A] = \bigcup \{X_n \colon n \in \mathbb{N}\}\),\\
so \(\exists a \in f[A] \colon a \in X_n\).\\
And since \(X_n\) is defined as \(X_n = \{X_{n-1} + f(X_{n-1}) \colon n \in \mathbb{Z}^+\}\), then \(a \in f(a)\).\\
Therefore, \(f(A) \subseteq f[A] \forall X \subseteq A\).
\end{proof}

The converse inclusion does not hold.

\begin{proof}
Let \(A = \{-1, 0, 1, 2\}, f(A) = \{a^2 \colon a \in A\}\).
\begin{align*}
A_0 &= A = \{-1, 0, 1, 2\} \\
A_1 &= A_0 \cup f(A_0) = \{-1,0,1,2\} \cup \{0, 1, 4\} = \{-1, 0 ,1, 2 ,4\}\\
\vdots
\end{align*}

Therefore, \(f(A) \subseteq f[A]\), but \(f[A] \not \subseteq f(A)\).
\end{proof}

\spacepls

{\large \textbf{c)} Show that when \(f(A) \subseteq A\), then \(f[A] = A\)}

\begin{proof}
Assume \(f(A) \subseteq A\).\\
By definition, \(f[A] = \bigcup \{A_n \colon n \in \mathbb{N}\}\),\\
which also defines \(A_0\) as \(A\) itself, and \(A_n = \{A_{n-1} \colon n \in \mathbb{Z}^+, a \in A\}\).\\
So, \(A \subseteq F[A]\).\\
And since \(f(A) \subseteq A\), and \(f[A] = \bigcup \{A_n \colon n \in \mathbb{N}\}\),\\
then the union of all \(A_n\) include all elements of \(A\),\\
and therefore, \(f[A] \subseteq A\).\\
Thus, \(f[A] = A\).
\end{proof}

\spacepls

{\large \textbf{d)} Show that for any partition of \(A\), the function \(f\) taking each element \(a \in A\) to its cell is a function on \(A\) into the powerset \(\wp (A)\) of \(A\) with the partition as its range.}

\begin{proof}
Let \(f(A) \colon A \to \wp (A)\), and \(A_i\) be any partition of \(A\).\\
Since \(\wp (A)\) contains all subsets of \(A\), then \(A_i \in \wp (A)\).\\
When applying \(f(A)\) in order to take each \(a \in A\) to its cell,
\(f(a)\) will assign to \(a \in A\) its correspondent partition, those in which \(a \in A_i\) is included.\\
Since \(a \in A_i\), and \(A_i \in \wp (A)\), then the range of \(f(a)\) for any \(a \in A_i\) would be any \(A_i \in \wp (A) \colon a \in A_i\)
\end{proof}

\spacepls

{\large \textbf{e)} Let \(f\colon A \to B\) be a function from \(A\) to \(B\). Recall the \textit{abstract inverse} \(f^{-1}\colon B \to \wp (A)\) by putting \(f^{-1}(b)= \{a \in A \colon f(a) = b\}\) for each \(b \in B\).}

\spacepls

\textbf{i)} Show that the collection of all sets \(f^{-1}(b)\) for \(b \in f(A) \subseteq B\) is a partition of \(A\).

\begin{proof}
Let \(f\colon A \to B\), and \(f^{-1}(b) = \{a \in A\colon f(a)=b\}\) for all \(b \in B\).\\
Now let \(S\) be the collection of \(f^{-1}(b) \, \forall \, b \in f(A) \subseteq B\).\\
Since \(a = f^{-1}(b)\), and \(a \in A\), then \(a \in S\).\\
And since \(f(a)\) is a proper function, \(\exists \, b \in B \forall \, a \in A\), so the union of all \(a \in S\) exhausts \(A\).\\
And because \(f(A)\) is a function, \(\exists ! \, a \in A \forall \, b \in B\), so all \(a \in S\) are pairwise disjoint.\\
Therefore, \(S\) is a partition, in a sense, of \(A\).
\end{proof}

\spacepls

\textbf{ii)} Is this still the case if we include in the collection the sets \(f^{-1}(b)\) for \(B \backslash f(A)\)?

\begin{proof}
Let \(S\) be our collection of \(f^{-1}(b) \colon b \in B \backslash f(A)\).\\
Since \(f(A) \colon A \to B\), then \(f(a) \in B\).\\
However, \(\nexists \, b \in B \colon b \in B \backslash f(A)\) and therefore \(S = \varnothing\).\\
And since a partition can't contain empty sets, then \(S\) is not a partition of \(A\).
\end{proof}

\section{Exercise 1.3.2}

{\large \textbf{a.i)} Is the floor function from \(\mathbb{R}\) into \(\mathbb{N}\) injective?}

No, because \(\forall \, r \in \mathbb{R} \colon 0 \leq r < 1\) will yield 0, and thus there are many \(r \in \mathbb{R}\) for each \(n \in \mathbb{N}\).

\spacepls

\textbf{ii)} Is it onto \(\mathbb{N}\)?\\
Yes, since \(\forall \, n \in \mathbb{N} \, \exists \, r \in \mathbb{R}\).

\spacepls

{\large \textbf{b)} Show that the composition of two bijections is a bijection.}

\begin{proof}
Let \(S \colon A \to B\) that is a bijection, and \(P \colon B \to A\) that is also a bijection.\\
Since \(S\) is a bijection, then it is also injective, which means that \(S^{-1}\) is a partial function from \(B\) to \(A\).\\
And since \(P\) is a bijection, then it is also injective, which means that \(P^{-1}\) is a partial function from \(A\) to \(B\).\\
Now let \((a,b) \in S\) and \((b,c) \in P\).\\
Since \(S \colon A \to B\), then \(a \in A\) and \(b \in B\).\\
Since \(P \colon B \to A\), then \(b \in B\) and \(c \in A\).\\
Therefore, \(P \circ S \colon A \to A\), and since \(\forall a \in A \, \exists c \in A\), then \(P \circ S\) is injective.\\
And since \(\forall c \in A \, \exists a \in A\), then \(P \circ S\) is onto.\\
Thus, \(P \circ S\) is a bijection.
\end{proof}

\pagebreak

{\large \textbf{c)} Use the equinumerosity principle to show that there is never any bijectiopn between a finite set and any of its proper subsets.}

\begin{proof}
Let \(A\) be a finite set, and \(B \subset A\).\\
Let \(f \colon A \to B\).\\
Since \(B \subset A\), then \(\#(A) \not = \#(B)\).\\
Therefore, by the equinumerosity principle, there's no \(f \colon A \to B\) that is a bijection between set \(A\) and its proper subset \(B\).
\end{proof}

\spacepls

{\large \textbf{d)} Give an example to show that there can be a bijection between an infinite set and certain of its proper subsets.}

\begin{proof}
Let \(S \colon \mathbb{N} \to \mathbb{Z}^+\) be the successor function.\\
For all natural numbers \(n \in \mathbb{N}\) (including 0), there is a positive integer \(z \in \mathbb{Z}^+\).\\
Therefore, \(S\) is an injective function.\\
Likewise, \(S\) is onto, since for all positive integers \(z \in \mathbb{Z}^+\) there's a natural number \(n \in \mathbb{N}\).\\
Hence, \(S\) is a bijection between an infinite set, \(\mathbb{N}\), and one of its proper subsets, \(\mathbb{Z}^+\).
\end{proof}

\spacepls

{\large \textbf{e)} Use the principle of comparison to show that for finite sets \(A,B\), if there are injective functions \(f \colon A \to B\) and \(g \colon B \to A\), then there is a bijection from \(A\) to \(B\).}

\begin{proof}
Let \(A,B\) be finite sets, and \(f \colon A \to B, g \colon B \to A\) that are both injective functions.\\
By definition, if \(f\) is injective, then \(\#(A) \leq \#(B)\).\\
And since \(g\) is injective, then \(\#(B) \leq \#(A)\).\\
Therefore, \(\#(A) = \#(B)\), and by the principle of equinumerosity, there is a bijection from \(A\) to \(B\).
\end{proof}

\section{Exercise 1.3.3}

{\large \textbf{a)} If a set \(A\) is partitioned into \(n\) cells, how many distinct elements of \(A\) need to bee selected to guarantee that at least two of them are in the same cell?}

\begin{proof}
Let \(A\) be a set partitioned into \(n\) cells.\\
The pigeonhole principle states that if \(\#(A) > k\#(B)\), then there is a \(b \in B \colon b = f(a)\) for at least \(k + 1\) distinct \(a \in A\).\\
So, if \(A\) is partitioned into \(n\) cells, then the cardinality of the set of all cells is \(n\).\\
And since we're looking for at least two elements \(a \in A\), then \(k+1 = 2\), therefore \(k = 1\).\\
This means that selecting at least \(n + 1\) (which is strictly greater than \(n\)) will guarantee at least two \(a \in A\) in the same cell.
\end{proof}

\spacepls

{\large \textbf{b)} Let \(K = \{1, 2, 3, 4, 5, 6, 7, 8\}\). How many distinct numbers must be selected from \(K\) to guarantee that there are two of them that sum to 9?}

\begin{proof}
Let \(A\) be the set of all unordered pairs \((x,y)\) with \(x,y \in K\), and \(x+y = 9\).\\
That means \(A = \{\{1,8\}, \{2,7\}, \{3,6\}, \{4,5\}\}\), which is a partition of \(K\) with \(4\) cells.\\
Since selecting any \(a \in A\) will ensure that a pair of \(k \in K\) will sum to 9, then we need to ensure getting at least 2 elements inside a cell of the partition set \(A\).\\
So, by the pigeonhole principle, if \(\#(K) > x\#(A)\), then there's an \(a \in A \colon a = f(k)\) for at least \(x+1\) distinct \(k \in K\).\\
\(\#(8) > 1\cdot \#(4)\), so \(x = 1\) and since we want 2 \(k \in K\) to be able to select any \(a \in A\),\\
we'll need to select \((\#(A) + 1) k \in K\), which is 5 elements.
\end{proof}

\section{Exercise 1.3.4}

{\large \textbf{a)} Let \(f \colon A \to B\) and \(g \colon B \to C\)}.

\textbf{i)} Show that if at least one of \(f,g\) is a constant function, then \(g \circ f \colon A \to C\) is a constant function.

\begin{proof}
Assume \(f \colon A \to B\) is a constant function.\\
So, \(\exists b \in B \, \forall a \in A \colon f(a) = b\).\\
This means that \(image(f(a)) = b\).\\
Now, the composition \(g \circ f\) means that function \(g\) will have its source as the image of \(f(a)\),\\
and because the image of \(f(a)\) is a constant, therefore the source of \(g\) is a constant.\\
Hence, \(g \circ f\) is also a constant function.
\end{proof}

\spacepls

\textbf{ii)} If \(g \circ f \colon A \to C\) is a constant function, does it follow that at least one of \(f,g\) is a constant function? Give a verification or a counterexample.

\spacepls

No. \(g \circ f\) could be a constant function even if both \(g,f\) are not. Consider the following counterexample:

\begin{proof}
Let \(f \colon \mathbb{R} \to \mathbb{R}\), defined by putting \(f(r) = \{-r \colon 0 \leq r < 1\}\),\\
and \(g \colon \mathbb{R} \to \mathbb{Z}\), defined as the ceiling function which yields the smallest integer greater or equal than \(r \in \mathbb{R}\).\\
By definition, function \(f\) gives the additive inverse of all entries \(r\) from 0 to 1, without including the 1.\\
Then, function \(g\) map each \(r\) from \(f(r)\) to a 0, since it is the smallest integer greater or equal to all real numbers included in \(f(r)\).\\
Although none of these functions are constant functions (since neither of them map all inputs to a single value in \(\mathbb{R}\) for \(f\) and in \(\mathbb{Z}\) for \(g\)),
the composition \(g \circ f\) does yield a single value for all inputs.\\
Therefore, \(g \circ f\) could be a constant function even when both \(g,f\) are not.
\end{proof}
\end{document}