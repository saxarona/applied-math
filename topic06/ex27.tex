\documentclass[titlepage, letterpaper, fleqn]{article}
\usepackage[utf8]{inputenc}
\usepackage{fancyhdr} % fancy headers, of course!
\usepackage{amsmath} % what do you think?
\usepackage{amsthm} % theorems!
\usepackage{extramarks} % more cute things
\usepackage{enumitem} % i'm not sure...
\usepackage{multicol} % multicolumn...?
\usepackage{amssymb} % more symbols
\usepackage{MnSymbol} % moar symbols?
\usepackage{booktabs} % cool looking tables
\usepackage{tikz} %venn and shizzle
\usepackage{tikz-qtree-compat} %tableaux
\usepackage{lipsum} %lorem ipsum dolor sit amet f u
\usepackage{mathrsfs} %math script for calligraphic scripting, I GUESS
\usepackage{listings}

\topmargin=-0.45in
\evensidemargin=0in
\oddsidemargin=0in
\textwidth=6.5in
\textheight=9.0in
\headsep=0.25in


%
% You should change this things~
%

\newcommand{\mahteacher}{Dr. Viacheslav Kalashnikov}
\newcommand{\mahclass}{Applied Mathematics}
\newcommand{\mahtitle}{Topic VI - Activity 27}
\newcommand{\mahdate}{November 30, 2016}
\newcommand{\spacepls}{\vspace{5mm}}
\newcommand{\until}{\mathscr{U}}
\renewcommand\qedsymbol{\(\blacksquare\)}

%
% Header markings
%

\pagestyle{fancy}
\lhead{1170065 - Xavier Sánchez}
\chead{}
\rhead{}
\lfoot{}
\rfoot{}


\renewcommand\headrulewidth{0.4pt}
\renewcommand\footrulewidth{0.4pt}

\setlength\parindent{0pt}


%
% Create Problem Sections (stolen directly from jdavis/latex-homework-template @ github!)
%

\newcommand{\enterProblemHeader}[1]{
\nobreak\extramarks{}{Problem \arabic{#1} continued on next page\ldots}\nobreak{}
\nobreak\extramarks{Problem \arabic{#1} (continued)}{Problem \arabic{#1} continued on next page\ldots}\nobreak{}
}

\newcommand{\exitProblemHeader}[1]{
\nobreak\extramarks{Problem \arabic{#1} (continued)}{Problem \arabic{#1} continued on next page\ldots}\nobreak{}
\stepcounter{#1}
\nobreak\extramarks{Problem \arabic{#1}}{}\nobreak{}
}

\setcounter{secnumdepth}{0}
\newcounter{partCounter}
\newcounter{homeworkProblemCounter}
\setcounter{homeworkProblemCounter}{1}
\nobreak\extramarks{Exercise \arabic{homeworkProblemCounter}}{}\nobreak{}

%Solution Environment
\newenvironment{solution}
{\renewcommand\qedsymbol{$\square$}\begin{proof}[Solution]}
{\end{proof}}

% Alias for the Solution section header
%\newcommand{\solution}{\textbf{\Large Solution}}

%Alias for the new step section
\newcommand{\steppy}[1]{\textbf{\large #1}}

%
% Homework Problem Environment
%
% This environment takes an optional argument. When given, it will adjust the
% problem counter. This is useful for when the problems given for your
% assignment aren't sequential. See the last 3 problems of this template for an
% example.
%
\newenvironment{homeworkProblem}[1][-1]{
\ifnum#1>0
\setcounter{homeworkProblemCounter}{#1}
\fi
\section{Exercise \arabic{homeworkProblemCounter}}
\setcounter{partCounter}{1}
\enterProblemHeader{homeworkProblemCounter}
}{
\exitProblemHeader{homeworkProblemCounter}
}

%
% My actual info
%

\title{
\vspace{1in}
\textbf{Tecnológico de Monterrey} \\
\vspace{0.5in}
\textmd{\mahclass} \\
\large{\textit{\mahteacher}} \\
\vspace{0.5in}
\textsc{\mahtitle}\\
\textsc{Elements of Optimization with Uncertainty}\\
\textsc{6.4.1, 6.4.2}\\
\author{01170065  - MIT \\
Xavier Fernando Cuauhtémoc Sánchez Díaz \\
\texttt{xavier.sanchezdz@gmail.com}}
\date{\mahdate}
}

\begin{document}

\begin{titlepage}
\maketitle
\end{titlepage}

%
% Actual document starts here~
%

\section{Exercise 6.4.1}

{\large Solve the 2-persons zero-sum matrix game with the payoff table given below.
Use the primal or dual simplex method after having reduced the game to a linear programming problem.

$$A =
\begin{bmatrix}
-2 & 3 & 0 & 4 \\
1 & -4 & 2 & -2 \\
5 & 0 & 3 & -1
\end{bmatrix}$$}

\begin{solution}
Assuming the player 1 is the row player, then we need to create $A^T$:
$$
A^T =
\begin{bmatrix}
-2 & 1 & 5 \\
3 & -4 & 0 \\
0 & 2 & 3 \\
4 & -2 & -1
\end{bmatrix}
$$

Then we need to convert it to a LP Problem.
Since we only have coefficients and no values,
we add another variable and include it as the value we're looking for.
After some algebra, the equations look like these:

\begin{align*}
	Z = v \to \max \\
	\text{subject to} \\
	-2x_1 + x_2 + 5x_3 - x_4 \geq 0 \\
	3x_1 + 4x_2 - x_4 \geq 0 \\
	2x_2 + 3x_3 - x_4 \geq 0 \\
	4x_1 - 2x_2 - x_3 - x_4 \geq 0
\end{align*}

And since we're looking for a probability distribution between actions $x_1, x_2, x_3$, then we add the constraint $x_1 + x_2 + x_3 = 1$.

Solving automatically in IORTutorial yields the following output:

\begin{lstlisting}[basicstyle=\tiny]
Linear Programming Model:
Number of Decision Variables:      4
Number of Functional Constraints:  5

Max Z =      0 X1 +    0 X2 +    0 X3 +    1 X4 

subject to

 1)         -2 X1 +    1 X2 +    5 X3 -    1 X4 >=       0
 2)          3 X1 -    4 X2 +    0 X3 -    1 X4 >=       0
 3)          0 X1 +    2 X2 +    3 X3 -    1 X4 >=       0
 4)          4 X1 -    2 X2 -    1 X3 -    1 X4 >=       0
 5)          1 X1 +    1 X2 +    1 X3 +    0 X4  =       1

and
        X1 >= 0, X2 >= 0, X3 >= 0.


Solve Automatically by the Simplex Method:

Optimal Solution                            |       Sensitivity Analysis
Objective Function: 1.5                     |    Objective Function Coefficients
 Variable   |   Value                       |     Current Value | Minimum | Maximum 
____________|___________                    |    _______________|_________|_________
X1          |0.5                            |    0              |-5       |3
X2          |0                              |    0              |-infin   |1
X3          |0.5                            |    0              |-1       |5
X4          |1.5                            |    1              |0        |infin
                                            |
                                            |
                                            |         Right Hand Sides
Constraint|Slack or Surplus|Shadow Price    |     Current Value | Minimum | Maximum 
__________|________________|____________    |    _______________|_________|_________
1         |0               |-0              |     0             |-infin   |0
2         |0               |-0              |     0             |-infin   |0
3         |0               |-0.625          |     0             |-0       |2.4
4         |0               |-0.375          |     0             |-0       |0
5         |0               |1.5             |     1             |0        |infin

x1*=0.5, x2*=0, x3*=0.5
Z=1.5
\end{lstlisting}

\end{solution}

\section{Exercise 6.4.2}

{\large Solve the 2-persons zero-sum matrix game with the payoff table given below.
Use the primal or dual simplex method after having reduced the game to a linear programming problem.

$$A =
\begin{bmatrix}
1 & -3 & 2 & -2 & 1 \\
2 & 3 & 0 & 3 & -2 \\
0 & 4 & -1 & -3 & 2 \\
-4 & 0 & -2 & 2 & -1
\end{bmatrix}$$}

\begin{solution}
Assuming the player 1 is the row player, then we need to create $A^T$:
$$
A^T =
\begin{bmatrix}
1 & 2 & 0 & -4 \\
-3 & 3 & 4 & 0 \\
2 & 0 & -1 & -2 \\
-2 & 3 & -3 & 2 \\
1 & -2 & 2 & -1
\end{bmatrix}
$$

Then we need to convert it to a LP Problem.
Since we only have coefficients and no values,
we add another variable and include it as the value we're looking for.
After some algebra, the equations look like these:

\begin{align*}
	Z = v \to \max \\
	\text{subject to} \\
	x_1 + 2x_2 -4x_4 - x_5 \geq 0 \\
	-3x_1 + 3x_2 + 4x_3 - x_5 \geq 0 \\
	2x_1 -x_3 -2x_4 - x_5 \geq 0 \\
	-2x_1 3x_2 -3x_3 + 2x_4 - x_5 \geq 0 \\
	x_1 -2x_2 + 2x_3 -x_4 -x_5 \geq 0
\end{align*}

And since we're looking for a probability distribution between actions $x_1, x_2, x_3, x_4$, then we add the constraint $x_1 + x_2 + x_3 + x_4 = 1$.

Solving automatically in IORTutorial yields the following output:
\begin{lstlisting}[basicstyle=\tiny]
Linear Programming Model:
Number of Decision Variables:      5
Number of Functional Constraints:  6

Max Z =      0 X1 +    0 X2 +    0 X3 +    0 X4 +    1 X5 

subject to

 1)          1 X1 +    2 X2 +    0 X3 -    4 X4 -    1 X5 >=       0
 2)         -3 X1 +    3 X2 +    4 X3 +    0 X4 -    1 X5 >=       0
 3)          2 X1 +    0 X2 -    1 X3 -    2 X4 -    1 X5 >=       0
 4)         -2 X1 +    3 X2 -    3 X3 +    2 X4 -    1 X5 >=       0
 5)          1 X1 -    2 X2 +    2 X3 -    1 X4 -    1 X5 >=       0
 6)          1 X1 +    1 X2 +    1 X3 +    1 X4 +    0 X5  =       1

and
        X1 >= 0, X2 >= 0, X3 >= 0, X4 >= 0.


Solve Automatically by the Simplex Method:

Optimal Solution                            |       Sensitivity Analysis
Objective Function: -0.019                  |    Objective Function Coefficients
 Variable   |   Value                       |     Current Value | Minimum | Maximum 
____________|___________                    |    _______________|_________|_________
X1          |0.31                           |    0              |-0.1     |0.172
X2          |0.266                          |    0              |-0.556   |0.071
X3          |0.209                          |    0              |-0.208   |0.091
X4          |0.215                          |    0              |-0.067   |1.25
X5          |-0.019                         |    1              |0        |infin
                                            |
                                            |
                                            |         Right Hand Sides
Constraint|Slack or Surplus|Shadow Price    |     Current Value | Minimum | Maximum 
__________|________________|____________    |    _______________|_________|_________
1         |0               |-0.013          |     0             |-0.487   |1.133
2         |0.722           |-0              |     0             |-infin   |0.722
3         |0               |-0.063          |     0             |-0.3     |0.278
4         |0               |-0.361          |     0             |-0.053   |2.882
5         |0               |-0.563          |     0             |-0.034   |2.333
6         |0               |-0.019          |     1             |0        |infin

\end{lstlisting}
\end{solution}

\end{document}