\documentclass{beamer}
\usepackage{graphicx, multirow, array, hyperref, listings}
\usepackage[utf8]{inputenc}
\usepackage{etoolbox}

% There are many different themes available for Beamer. A comprehensive
% list with examples is given here:
% http://deic.uab.es/~iblanes/beamer_gallery/index_by_theme.html
% You can uncomment the themes below if you would like to use a different
% one:
%\usetheme{AnnArbor}
%\usetheme{Antibes}
%\usetheme{Bergen}
%\usetheme{Berkeley}
%\usetheme{Berlin}
\usetheme{Boadilla}
%\usetheme{boxes}
%\usetheme{CambridgeUS}
%\usetheme{Copenhagen}
%\usetheme{Darmstadt}
%\usetheme{default}
%\usetheme{Frankfurt}
%\usetheme{Goettingen}
%\usetheme{Hannover}
%\usetheme{Ilmenau}
%\usetheme{JuanLesPins}
%\usetheme{Luebeck}
%\usetheme{Madrid}
%\usetheme{Malmoe}
%\usetheme{Marburg}
%\usetheme{Montpellier}
%\usetheme{PaloAlto}
%\usetheme{Pittsburgh}
%\usetheme{Rochester}
%\usetheme{Singapore}
%\usetheme{Szeged}
%\usetheme{Warsaw}

\usecolortheme{rose}
% \makeatletter
% \patchcmd{\beamer@sectionintoc}
%   {\vfill}
%   {\vskip 2ex}
%   {}
%   {}
% \makeatother

\setbeamertemplate{sections/subsections in toc}[sections numbered]

\hypersetup{
  colorlinks=true,
  linkcolor=blue,
  urlcolor=cyan
}
%\parskip=1.5ex

\title{Symmetric Eigenvalues Problems}

% A subtitle is optional and this may be deleted
\subtitle{Final Presentation}

\author[X. S\'{a}nchez]{Xavier~S\'{a}nchez~D\'{i}az\\1170065}
% - Give the names in the same order as the appear in the paper.
% - Use the \inst{?} command only if the authors have different
%   affiliation.

\institute[ITESM] % (optional, but mostly needed)
{
Applied Mathematics
}
% - Use the \inst command only if there are several affiliations.
% - Keep it simple, no one is interested in your street address.

\date{\today}
% - Either use conference name or its abbreviation.
% - Not really informative to the audience, more for people (including
%   yourself) who are reading the slides online

\subject{Applied Mathematics}
% This is only inserted into the PDF information catalog. Can be left
% out. 

% If you have a file called "university-logo-filename.xxx", where xxx
% is a graphic format that can be processed by latex or pdflatex,
% resp., then you can add a logo as follows:

% \pgfdeclareimage[height=0.5cm]{university-logo}{university-logo-filename}
% \logo{\pgfuseimage{university-logo}}

% Delete this, if you do not want the table of contents to pop up at
% the beginning of each subsection:
% \AtBeginSubsection[]
% {
%   \begin{frame}<beamer>{Outline}
%     \tableofcontents[currentsection,currentsubsection]
%   \end{frame}
% }


% Let's get started
\begin{document}

\begin{frame}
\titlepage
\end{frame}



\begin{frame}[allowframebreaks=1]{Outline}
\tableofcontents[hideallsubsections=true]
  % You might wish to add the option [pausesections]
\end{frame}



% Section and subsections will appear in the presentation overview
% and table of contents.
\section{Eigenvalues and Eigenvectors}

\subsection{Eigenvalues}

\begin{frame}{Eigenvalues}{Eigenvalues and Eigenvectors}

\begin{definition}
The characteristic polynomial of a square matrix $A$ is

\[p(x) = \text{det}(A - xI)\]

The roots of the characteristic polynomial are the \textit{eigenvalues} of $A$.
\end{definition}
\end{frame}

\begin{frame}{More about eigenvalues}{Eigenvalues and Eigenvectors}
An $n$-by-$n$ matrix has $n$ eigenvalues. If these eigenvalues are real, they are usually indexed from largest to smallest:
\[\lambda_n(A) \leq \dots \leq \lambda_1\]
\pause
If $X$ is a square matrix and it is non-singular (i.e. it has an inverse matrix), and $B = X^{-1} AX$, then $A$ and $B$ are similar.
\pause

If two matrices are similar they have \alert{exactly} the same eigenvalues.
\end{frame}

\subsection{Eigenvectors}

\begin{frame}{Eigenvectors}{Eigenvalues and Eigenvectors}
\begin{definition}
If $\lambda \in \lambda (A)$, then there exists a nonzero vector $x$ so that $Ax = \lambda x$. This $x$ vector is said to be an \textit{eigenvector} for $A$ associated with $\lambda$.
\end{definition}
\end{frame}

\section{QR Factorization}
\begin{frame}{QR Factorization}{Definition}
\begin{definition}
If $A \in \mathbb{R}^{m \times n}$, then there exists an orthogonal $Q$ which is real and square with $m$ rows, and an upper triangular matrix $R$ with $m$ rows and $n$ columns so that

\[A = QR\]
\end{definition}
\pause

Recall that $Q$ is orthogonal if $Q^T Q = I$.
\end{frame}

\section{Symmetric Eigenvalues Problem}

\begin{frame}{Description}{Symmetric Eigenvalues Problems}
Intel offers the following description in the LAPACK Computational Routines webpage:

\begin{definition}
Given an $n$-by-$n$ real symmetric or complex Hermitian matrix $A$, find the eigenvalues $\lambda$ and the corresponding eigenvectors $x$ that satisfy the equation $Ax = \lambda x$.
\end{definition}
\pause

These problems come in different `flavors'.
\end{frame}

\begin{frame}{Symmetric-Definite Generalized Eigenproblem}{Symmetric Eigenvalues Problems}

\begin{block}{Symmetric-definite problem}
\[Ax = \lambda B x\]
for $A \in \mathbb{R}^{n \times n}$ which is symmetric, and $B \in \mathbb{R}^{n \times n}$ which is symmetric positive definite.
\end{block}

\pause

Recall that a matrix $A \in \mathbb{R}^{n \times n}$ is positive definite if $x^T Ax > 0$ for all nonzero $x$ (which is a real vector).
\end{frame}

\begin{frame}{Generalized Singular Value Problem}{Symmetric Eigenvalues Problems}
\begin{block}{Generalized singular value problem}
\[A^T Ax = \mu^2 B^T Bx\]
for $A \in \mathbb{R}^{m_1 \times n}$ and $B \in \mathbb{R}^{m_2 \times n}$.
\end{block}
\pause
These problems are generalizations of the symmetric eigenvalue problem and the singular value problem respectively.
\end{frame}

\subsection{Properties}

\begin{frame}{Some properties}{Symmetric Eigenvalue Problems}
Symmetry guarantees that all of $A$'s eigenvalues are real, and that there is an orthonormal basis of eigenvectors.
\pause
Remember a set of vectors is orthonormal if it is not orthogonal, i.e. $x_i^T xj = \delta_{ij}$
\pause

\begin{theorem}
If $A \in \mathbb{R}^{n \times n}$ is symmetric, then there exists a real orthogonal $Q$ such that
\[Q^T A Q = \Lambda = \text{diag}(\lambda_1 , \dots , \lambda_n)\]
\end{theorem}

\pause

The function \texttt{diag()} means a diagonal matrix, i.e. with both upper and lower bandwidth zero.
\end{frame}

\begin{frame}{Law of Inertia}{Symmetric Eigenvalue Problems}
The \textit{inertia} of a symmetric matrix $A$ is a triplet of nonnegative integers $(m, z, p)$ where $m$, $z$ and $p$ are respectively the numbers of negative, zero and positive eigenvalues.
\pause

\subsection{Law of Inertia}

\begin{theorem}{Sylvester Law of Inertia}

If $A \in \mathbb{R}^{n \times n}$ is symmetric and $X \in \mathbb{R}^{n \times n}$ is non-singular, then $A$ and $X^T A X$ have the same inertia.
\end{theorem}
\pause
Recall that a non-singular matrix always has an inverse.
\end{frame}

\section{Power Iterations}

\begin{frame}[fragile]{What is a power iteration?}{Power Iterations}
\begin{definition}
A power iteration is an eigenvalue algorithm that will produce the a number $\lambda$, which is the greatest eigenvalue of $A$, and a nonzero vector $x$ (the corresponding eigenvector) such that
\[A x = \lambda x\]
\end{definition}
\pause
\begin{lstlisting}[mathescape=true]
for $k=1,2,\dots$
  $z^{(k)} = Aq^{(k-1)}$
  $q^{(k)} = z^{(k)}/\Vert z^{(k)} \Vert _2$
  $\lambda^{(k)} = [q^{(k)}]^T Aq^{(k)}$
end
\end{lstlisting}
\end{frame}

\begin{frame}[fragile]{Rayleigh Quotient Iteration}{Power Iterations}
Based on the fact that
\[\lambda = r(x) \equiv \dfrac{x^T Ax}{x^T x}\]
\pause
we have the algorithm:
\begin{lstlisting}[mathescape=true]
for $k=0,1,\dots$
  $\mu_l = r(x_k)$
  Solve $(A - \mu_k I) z_{k+1} = x_k$ for $z_{k+1}$
  $x_{k+1} = z_{k+1}/ \Vert z_{k+1}\Vert_2$
end
\end{lstlisting}
\end{frame}

\section{QR and Tridiag}

\begin{frame}{Reduction to Tridiagonal}{QR and Tridiag}
If $A$ is symmetric, then it is possible to find an orthogonal $Q$ such that
\[Q^T A Q = T\]
 is tridiagonal. This is called the \alert{tridiagonal decomposition}.
\end{frame}

\begin{frame}{QR Factorization}{QR and Tridiag}
Recall that if $A \in \mathbb{R}^{m \times n}$, then there exists an orthogonal $Q$ which is real and square with $m$ rows, and an upper triangular $R$ with $m$ rows and $n$ columns so that $A = QR$.

\pause

Also remember that a matrix $T$ is tridiagonal if it has the following format:

\[
\left[
\begin{array}{cccc}
a & b & 0 & 0 \\
c & d & e & 0 \\
0 & f & g & h \\
0 & 0 & i & j
\end{array}
\right]
\]

\end{frame}

\begin{frame}{QR Iteration and tridiagonal matrices}{QR and Tridiag}
Some facts pertaining to the $QR$ iteration:

\begin{itemize}
  \item \textit{Preservation of Form}. If $T = QR$ is the $QR$ factorization of a symmetric tridiagonal matrix, then $Q$ has lower bandwidth 1 and $R$ has upper bandwidth 2. $Q^T T Q$ is also symmetric and tridiagonal.

  \item \textit{Shifts}. If $s$ is a real number and $T - sI = QR$ is the QR factorization, then $RQ + sI = Q^TTQ$ is also tridiagonal.
\end{itemize}
\end{frame}

\subsection{More on Tridiags}

\begin{frame}[fragile]{Eigenvalues by Bisection}{More on Tridiagonals}
If $T_r$ denote the leading $r$-by-$r$ principal submatrix of the matrix $T$, then its polynomial is $p_r(x) = \texttt{det}(T_r - xI)$.
\pause
Therefore, the following algorithm can calculate an approximation of the eigenvalue of $T$:
\begin{lstlisting}[mathescape=true]
while $|y - z| > \texttt{tol}\cdot (|y|+|z|)$
  $x = (y + z)/2$
  if $p_n(x) \cdot p_n(y) <0$
    $z = x$
  else
    $y = x$
  end
end
\end{lstlisting}
\texttt{tol} is a small tolerance constant which is positive.
\end{frame}

\begin{frame}[fragile]{Sturm sequence}{More on Tridiagonals}
Sometimes it is necessary to compute the $k$th largest eigenvalue of $T$
for some prescribed value of $k$.
By looking at sign changes in the strict order of the set of eigenvalues of $T_r$ one can get to the following algorithm:

\begin{lstlisting}[mathescape=true]
while $|z - y| > \texttt{u}(|y|+|z|)$
  $x = (y + z)/2$
  if $a(x) \geq n - k$
    $z = x$
  else
    $y = x$
  end
end
\end{lstlisting}
where $a(\lambda)$ denotes the number of sign changes in the sequence $\{p_0(\lambda),\dots, p_n(\lambda)\}$
\end{frame}

\section{Jacobi Methods}
\begin{frame}{Description}{Jacobi Method}
Jacobi methods for the Symmetric Eigenvalue Problem work by performing a sequence of orthogonal similarity updates $A \leftarrow Q^T A Q$ with the property that each new $A$, although full, is more diagonal than the previous one.
Eventually, the off-diagonal entries are small enough to be declared as zero.
\end{frame}

\begin{frame}[fragile]{Classic Jacobi}{Jacobi Method}
Rows and columns ($p,q$) are altered when the subproblem is solved. $p,q$ are the rows and columns of the main submatrix  of $A$.
\pause
\begin{lstlisting}[mathescape=true]
$V = I_n, \delta = \texttt{tol} \cdot \Vert A \Vert_F$
while off($A) > \delta $
  Choose ($p,q$) so $|a_{pq}| = \text{max}_{i\not = j} |a_{ij}|$
  [$c,s$] = \texttt{symSchur2($A, p, q$)}
  $A = J(p,q,\theta)^T A J(p,q,\theta)$
  $V = V J(p,q,\theta)$
end
\end{lstlisting}
Where $\Vert A \Vert_F$ is the Frobenius norm of $A$, and \texttt{symShcur2} is a function that generates the 2-by-2 symmetric Schur Decomposition.
\end{frame}

\begin{frame}{Symmetric Schur Decomposition}{Jacobi}
\begin{definition}
If $A \in \mathbb{R}^{n \times n}$ is symmetric, then there exists an orthogonal $Q$ so that
\[Q^T A Q = \text{diag}(\lambda_1, \dots, \lambda_n)\]
\end{definition}
\end{frame}

\section[SVD]{Single Value Decomposition}
\begin{frame}{Description}{Single Value Decomposition}
The single value decomposition (SVD) is an orthogonal matrix reduction.

If $A$ is a real $m$-by-$n$ matrix, then there exists orthogonal matrices $U$ and $V$ (both square, $m$-by-$m$ and $n$-by-$n$ respectively) such that
\[U^T A V = diag(\sigma_1 , \dots , \sigma_p) \in \mathbb{R}^{m \times n}, p = \text{min}{m,n}\]

where $\sigma_1 \geq \sigma_2 \geq \dots \geq \sigma_p \geq 0$.
\end{frame}

\begin{frame}{The SVD Algorithm}{Single Value Decomposition}
A variant of the $QR$ algorithm can be used to compute the SVD of an $A \in \mathbb{R}^{m \times n}$ with $m \geq n$:

\begin{itemize}
  \item \textit{Step 1}. Form $C = A^T A$,
  \item \textit{Step 2}. Use the symmetric QR algorithm to compute $V_1^T C V_1 = diag(\sigma_i^2)$.
  \item \textit{Step 3}. Apply $QR$ with column pivoting to $AV_1$ obtaining $U^T(AV_1)\Pi = R$
\end{itemize}
\pause
Recall that $\Pi$ is a permutation matrix.
\end{frame}

\section{Quadratic Eigenvalue Problem}

\begin{frame}{Description}{Quadratic Eigenvalue Problem}
There are some more complex symmetric eigenvalue problems.

\begin{example}
\[(\lambda^2 M + \lambda C + K) x = 0\]
where $M, C, K, \in \mathbb{R}^{n \times n}$
\end{example}
\pause
The eigenvalue solves the following equation:
\[\lambda = \dfrac{-(x^H C x) \pm \sqrt{(x^H C x)^2 -4(x^H Mx)(x^H Kx)}}{2(x^H Mx)}\]
Assuming that $x^H M x \not = 0$.
\pause

Recall that $x^H$ is the conjugate transpose of a vector $x$.
\end{frame}

\begin{frame}
\vspace{4ex}
{\Huge Thank you!}
%\vspace{2ex}
%Slides available at \url{https://dl.dropboxusercontent.com/u/15350213/symm_eigen.pdf}

\end{frame}
\end{document}