\documentclass[titlepage, letterpaper, fleqn]{article}
\usepackage[utf8]{inputenc}
\usepackage{fancyhdr} % fancy headers, of course!
\usepackage{amsmath} % what do you think?
\usepackage{amsthm} % theorems!
\usepackage{extramarks} % more cute things
\usepackage{enumitem} % i'm not sure...
\usepackage{multicol} % multicolumn...?
\usepackage{amssymb} % more symbols
\usepackage{MnSymbol} % moar symbols?
\usepackage{booktabs} % cool looking tables
\usepackage{tikz} %venn and shizzle
\usepackage{tikz-qtree-compat} %tableaux
\usepackage{lipsum} %lorem ipsum dolor sit amet f u
\usepackage{mathrsfs} %math script for calligraphic scripting, I GUESS

\topmargin=-0.45in
\evensidemargin=0in
\oddsidemargin=0in
\textwidth=6.5in
\textheight=9.0in
\headsep=0.25in


%
% You should change this things~
%

\newcommand{\mahteacher}{Dr. Viacheslav Kalashnikov}
\newcommand{\mahclass}{Applied Mathematics}
\newcommand{\mahtitle}{Topic III - Activity 16}
\newcommand{\mahdate}{November 09, 2016}
\newcommand{\spacepls}{\vspace{5mm}}
\newcommand{\until}{\mathscr{U}}
\renewcommand\qedsymbol{\(\blacksquare\)}

%
% Header markings
%

\pagestyle{fancy}
\lhead{1170065 - Xavier Sánchez}
\chead{}
\rhead{}
\lfoot{}
\rfoot{}


\renewcommand\headrulewidth{0.4pt}
\renewcommand\footrulewidth{0.4pt}

\setlength\parindent{0pt}


%
% Create Problem Sections (stolen directly from jdavis/latex-homework-template @ github!)
%

\newcommand{\enterProblemHeader}[1]{
\nobreak\extramarks{}{Problem \arabic{#1} continued on next page\ldots}\nobreak{}
\nobreak\extramarks{Problem \arabic{#1} (continued)}{Problem \arabic{#1} continued on next page\ldots}\nobreak{}
}

\newcommand{\exitProblemHeader}[1]{
\nobreak\extramarks{Problem \arabic{#1} (continued)}{Problem \arabic{#1} continued on next page\ldots}\nobreak{}
\stepcounter{#1}
\nobreak\extramarks{Problem \arabic{#1}}{}\nobreak{}
}

\setcounter{secnumdepth}{0}
\newcounter{partCounter}
\newcounter{homeworkProblemCounter}
\setcounter{homeworkProblemCounter}{1}
\nobreak\extramarks{Exercise \arabic{homeworkProblemCounter}}{}\nobreak{}

%Solution Environment
\newenvironment{solution}
{\renewcommand\qedsymbol{$\square$}\begin{proof}[Solution]}
{\end{proof}}

% Alias for the Solution section header
%\newcommand{\solution}{\textbf{\Large Solution}}

%Alias for the new step section
\newcommand{\steppy}[1]{\textbf{\large #1}}

%
% Homework Problem Environment
%
% This environment takes an optional argument. When given, it will adjust the
% problem counter. This is useful for when the problems given for your
% assignment aren't sequential. See the last 3 problems of this template for an
% example.
%
\newenvironment{homeworkProblem}[1][-1]{
\ifnum#1>0
\setcounter{homeworkProblemCounter}{#1}
\fi
\section{Exercise \arabic{homeworkProblemCounter}}
\setcounter{partCounter}{1}
\enterProblemHeader{homeworkProblemCounter}
}{
\exitProblemHeader{homeworkProblemCounter}
}

%
% My actual info
%

\title{
\vspace{1in}
\textbf{Tecnológico de Monterrey} \\
\vspace{0.5in}
\textmd{\mahclass} \\
\large{\textit{\mahteacher}} \\
\vspace{0.5in}
\textsc{\mahtitle}\\
\textsc{3.6.1 Random variables and expected values}\\
\textsc{3.6.2 Random variables and expected values}\\
\textsc{3.6.3 Random variables and expected values}\\
\author{01170065  - MIT \\
Xavier Fernando Cuauhtémoc Sánchez Díaz \\
\texttt{mail@gmail.com}}
\date{\mahdate}
}

\begin{document}

\begin{titlepage}
\maketitle
\end{titlepage}

%
% Actual document starts here~
%

\section{Exercise 3.6.1}

{\large \textbf{a)} A fair die is rolled once.
Let the random variable \(X\) equal the value that comes up.
Find the expected value of \(X, E(X)\).}

\begin{solution}
Since all six numbers are equally likely to come up, then the probability of each is \(\frac{1}{6}\).
The expected value of \(X\) is then \(E(X) = \frac{1}{6}(1 + 2 + 3 + 4 + 5 + 6) = \frac{21}{6} = 3.5\).
\end{solution}

\spacepls

{\large \textbf{b)} The die is now \textit{loaded} so that a 2 comes up twice as often than any other number.
Find the new expected value of \(X\).}

\begin{solution}
The new expected value of \(X\) is now calculated as \(E(X) = \frac{1}{7}(1 + 3 + 4 + 5 + 6) + \frac{2}{7} \cdot 2 = \frac{19}{7} + \frac{4}{7} = \frac{23}{7} \approx 3.2857\).
\end{solution}

\spacepls

{\large \textbf{c)} Your answer to part \textbf{b)} should be greater or less than your answer to part \textbf{a)}? Explain why.}

\begin{solution}
The answer to part \textbf{b)} of the exercise should be lower because 2 is in the \textit{lower} part of the spectrum of available numbers on a die.
If the loaded face were 5 or 6, then the expected value would be greater.
Let's not forget that the expected value is a weighted average, and thus increasing the probability of a value in the lower threshold of the list of values will decrease it.
\end{solution}

\section{Exercise 3.6.2}

{\large Two fair dices are rolled.
The sample space \(S\) contains the 36 combinations of two numbers.
For each member \((r, s)\) of \(S\),
the random variable \(X(r,s) = r + s\).

\textbf{a)} Write a table showing the values for \(X\), and the probability of those values; instead of 36 columns each with probability \(\frac{1}{36}\),
do a column for each distinct value of \(X\) and show the probability of that value.}

% Please add the following required packages to your document preamble:
% \usepackage{booktabs}
\begin{table}[h!]
\centering
\begin{tabular}{@{}cccccccccccc@{}}
\toprule
X & 2 & 3 & 4 & 5 & 6 & 7 & 8 & 9 & 10 & 11 & 12 \\ \midrule
36 * P(X) & 1 & 2 & 3 & 4 & 5 & 6 & 5 & 4 & 3 & 2 & 1 \\ \bottomrule
\end{tabular}
\caption{Number of occurrences per value of \(X\)}
\label{tab01}
\end{table}

\spacepls

{\large \textbf{b)} Find the expected value of the sum of the numbers that come up when two fair dice are rolled.}

\begin{solution}
The expected value of \(X\) is calculated as follows:
\begin{align*}
E(X) & = \frac{1}{36}(2 + 12) + \frac{2}{36}(3 + 11) + \frac{3}{36}(4 + 10) + \frac{4}{36}(5 + 9) + \frac{5}{36}(6 + 8) + \frac{6}{36} \cdot 7 \\[2ex]
& = 14 \left( \frac{1}{36} + \frac{2}{36} + \frac{3}{36} + \frac{4}{36} + \frac{5}{36}\right) + \frac{6}{36} \cdot 7 \\[2ex]
& = \frac{35}{6} + \frac{7}{6} = \frac{42}{6}\\
& = 7 \qedhere
\end{align*}
\end{solution}

\section{Exercise 3.6.3}

{\large At a gambling casino, a ball will be drawn from a bin containing 43 red balls, 27 green balls, and 8 blue balls.
A player marks a game card with the color he or she believes will be picked.
The prize money for guessing the correct color is:

Red \$ 3.00

Green \$ 6.00

Blue \$ 10.00

The price of the game card is \$5.00.
Find the expected value of the prize money.}

\begin{solution}
The expected value is calculated as usual:

\begin{align*}
E(X) & = P(R) \cdot \$ 3.00 + P(G) \cdot \$ 6.00 + P(B) \cdot \$ 10.00 - \$ 5.00 \\
& = \frac{43}{78} \cdot \$ 3.00 + \frac{27}{78} \cdot \$ 6.00 + \frac{8}{78} \cdot \$ 10.00 - \$ 5.00 \\[2ex]
& = \frac{371}{78} - 5 \\[2ex]
& \approx -0.2435 \qedhere
\end{align*}
\end{solution}
\end{document}