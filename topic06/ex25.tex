\documentclass[titlepage, letterpaper, fleqn]{article}
\usepackage[utf8]{inputenc}
\usepackage{fancyhdr} % fancy headers, of course!
\usepackage{amsmath} % what do you think?
\usepackage{amsthm} % theorems!
\usepackage{extramarks} % more cute things
\usepackage{enumitem} % i'm not sure...
\usepackage{multicol} % multicolumn...?
\usepackage{amssymb} % more symbols
\usepackage{MnSymbol} % moar symbols?
\usepackage{booktabs} % cool looking tables
\usepackage{tikz} %venn and shizzle
\usepackage{tikz-qtree-compat} %tableaux
\usepackage{lipsum} %lorem ipsum dolor sit amet f u
\usepackage{mathrsfs} %math script for calligraphic scripting, I GUESS
\usepackage{listings}

\topmargin=-0.45in
\evensidemargin=0in
\oddsidemargin=0in
\textwidth=6.5in
\textheight=9.0in
\headsep=0.25in


%
% You should change this things~
%

\newcommand{\mahteacher}{Dr. Viacheslav Kalashnikov}
\newcommand{\mahclass}{Applied Mathematics}
\newcommand{\mahtitle}{Topic VI - Activity 25}
\newcommand{\mahdate}{November 30, 2016}
\newcommand{\spacepls}{\vspace{5mm}}
\newcommand{\until}{\mathscr{U}}
\renewcommand\qedsymbol{\(\blacksquare\)}

%
% Header markings
%

\pagestyle{fancy}
\lhead{1170065 - Xavier Sánchez}
\chead{}
\rhead{}
\lfoot{}
\rfoot{}


\renewcommand\headrulewidth{0.4pt}
\renewcommand\footrulewidth{0.4pt}

\setlength\parindent{0pt}


%
% Create Problem Sections (stolen directly from jdavis/latex-homework-template @ github!)
%

\newcommand{\enterProblemHeader}[1]{
\nobreak\extramarks{}{Problem \arabic{#1} continued on next page\ldots}\nobreak{}
\nobreak\extramarks{Problem \arabic{#1} (continued)}{Problem \arabic{#1} continued on next page\ldots}\nobreak{}
}

\newcommand{\exitProblemHeader}[1]{
\nobreak\extramarks{Problem \arabic{#1} (continued)}{Problem \arabic{#1} continued on next page\ldots}\nobreak{}
\stepcounter{#1}
\nobreak\extramarks{Problem \arabic{#1}}{}\nobreak{}
}

\setcounter{secnumdepth}{0}
\newcounter{partCounter}
\newcounter{homeworkProblemCounter}
\setcounter{homeworkProblemCounter}{1}
\nobreak\extramarks{Exercise \arabic{homeworkProblemCounter}}{}\nobreak{}

%Solution Environment
\newenvironment{solution}
{\renewcommand\qedsymbol{$\square$}\begin{proof}[Solution]}
{\end{proof}}

% Alias for the Solution section header
%\newcommand{\solution}{\textbf{\Large Solution}}

%Alias for the new step section
\newcommand{\steppy}[1]{\textbf{\large #1}}

%
% Homework Problem Environment
%
% This environment takes an optional argument. When given, it will adjust the
% problem counter. This is useful for when the problems given for your
% assignment aren't sequential. See the last 3 problems of this template for an
% example.
%
\newenvironment{homeworkProblem}[1][-1]{
\ifnum#1>0
\setcounter{homeworkProblemCounter}{#1}
\fi
\section{Exercise \arabic{homeworkProblemCounter}}
\setcounter{partCounter}{1}
\enterProblemHeader{homeworkProblemCounter}
}{
\exitProblemHeader{homeworkProblemCounter}
}

%
% My actual info
%

\title{
\vspace{1in}
\textbf{Tecnológico de Monterrey} \\
\vspace{0.5in}
\textmd{\mahclass} \\
\large{\textit{\mahteacher}} \\
\vspace{0.5in}
\textsc{\mahtitle}\\
\textsc{Elements of Combinatorial Optimization}\\
\textsc{6.2.1, 6.2.2}\\
\author{01170065  - MIT \\
Xavier Fernando Cuauhtémoc Sánchez Díaz \\
\texttt{xavier.sanchezdz@gmail.com}}
\date{\mahdate}
}

\begin{document}

\begin{titlepage}
\maketitle
\end{titlepage}

%
% Actual document starts here~
%

\section{Exercise 6.2.1}

{\large Four freight vessels are used to carry goods from one port to the other four ports (numerated as 1, 2, 3, 4).
Any vessel can be used to fulfill any of the four transportations.
However, given some differences among the ships and the freights,
the total transportation cost is much different for distinct combinations of ships and ports.
These costs are in the following table:}

\begin{table}[h!]
\centering
\begin{tabular}{@{}ccccc@{}}
\toprule
P/S & A & B & C & D \\ \midrule
1   & 5 & 4 & 6 & 7 \\
2   & 6 & 6 & 7 & 5 \\
3   & 7 & 5 & 7 & 6 \\
4   & 5 & 4 & 6 & 6 \\ \bottomrule
\end{tabular}
\caption{Ships (numbers) and their cost to all ports (letters)}
\label{tab:6.2.1}
\end{table}

{\large Formulate and solve this problem as an assignment problem using the Hungarian method.}

\spacepls

\begin{solution}
This problem can be solved easily using the IORTutorial via the Hungarian method since the cost matrix is square.
The results can be found below.

\begin{lstlisting}[basicstyle=\tiny]
Solve an Assignment Problem Interactively:
Number of tasks:    4

Step :
Subtract the smallest number in each row from every number in the row.
Enter the results in a new table.

                       Task
  Assignee| A      B      C      D      | Row Min
__________|_____________________________|________
     1    | 5      4      6      7      | 4
     2    | 6      6      7      5      | 5
     3    | 7      5      7      6      | 5
     4    | 5      4      6      6      | 4
          |                             | 

Step :
Subtract the smallest number in each column of the new table from every number in the column.

                       Task
  Assignee| A      B      C      D      |
__________|_____________________________|________
     1    | 1      0      2      3      | 
     2    | 1      1      2      0      | 
     3    | 2      0      2      1      | 
     4    | 1      0      2      2      | 
Col Min   | 1      0      2      0      | 

Step :
Enter the results in a new table.


                       Task
  Assignee| A      B      C      D      |
__________|_____________________________|________
     1    | 0      0      0      3      | 
     2    | 0      1      0      0      | 
     3    | 1      0      0      1      | 
     4    | 0      0      0      2      | 
          |                             | 

Step :
Determine the minimum number of lines needed to cross out all zeros.

                       Task
  Assignee| A      B      C      D      |
__________|_____________________________|________
          | |      |      |             |
     1    | 0      0      0      3      |
          | |      |      |             |
     2    |-0------1------0------0------|
          | |      |      |             |
     3    | 1      0      0      1      |
          | |      |      |             |
     4    | 0      0      0      2      |
          | |      |      |             |



Make the assignments.

                       Task
  Assignee| A      B      C      D      |
__________|_____________________________|________
     1    | 0      0      0      3      | 
     2    | 0      1      0      0      | 
     3    | 1      [0]    0      1      | 
     4    | 0      0      [0]    2      | 

Task A is assigned to Assignee 1
Task D is assigned to Assignee 2
Task B is assigned to Assignee 3
Task C is assigned to Assignee 4

Ship1 goes to port A
Ship2 goes to port D
Ship3 goes to port B
Ship4 goes to port C

Total minimum cost = 5 + 5 + 5 + 6 = 21
\end{lstlisting}
\end{solution}

\section{Exercise 6.2.2}
{\large Five employees are available to perform four jobs.
The time it takes each person to perform each job is given in the following table.
A dash indicates that said person can't do that particular job.
Determine the assignment of Employees to jobs that minimizes the total time required to perform the four jobs.}

\begin{table}[h!]
\centering
\begin{tabular}{@{}ccccc@{}}
\toprule
P/J & A  & B   & C  & D  \\ \midrule
1   & 22 & 418 & 30 & 18 \\
2   & 18 & -   & 27 & 22 \\
3   & 26 & 20  & 28 & 28 \\
4   & 16 & 22  & -  & 6  \\
5   & 21 & -   & 25 & 28 \\ \bottomrule
\end{tabular}
\caption{Employees (numbers) and their cost for performing jobs (letters)}
\label{tab:6_2_2}
\end{table}

\begin{solution}
The result can be viewed in the following code:
\begin{lstlisting}[basicstyle=\tiny]



Solve an Assignment Problem Interactively:
Number of tasks:    5

Step :
Subtract the smallest number in each row from every number in the row.
Enter the results in a new table.

                          Task
  Assignee| A      B      C      D      E      | Row Min
__________|____________________________________|________
     1    | 22     18     30     18     999    | 18
     2    | 18     999    27     22     999    | 18
     3    | 26     20     28     28     999    | 20
     4    | 16     22     999    14     999    | 14
     5    | 21     999    25     28     999    | 21
          |                                    | 

Step :
Subtract the smallest number in each column of the new table from every number in the column.

                          Task
  Assignee| A      B      C      D      E      |
__________|____________________________________|________
     1    | 4      0      12     0      981    | 
     2    | 0      981    9      4      981    | 
     3    | 6      0      8      8      979    | 
     4    | 2      8      985    0      985    | 
     5    | 0      978    4      7      978    | 
Col Min   | 0      0      4      0      978    | 

Step :
Enter the results in a new table.


                          Task
  Assignee| A      B      C      D      E      |
__________|____________________________________|________
     1    | 4      0      8      0      3      | 
     2    | 0      981    5      4      3      | 
     3    | 6      0      4      8      1      | 
     4    | 2      8      981    0      7      | 
     5    | 0      978    0      7      0      | 
          |                                    | 

Step :
Determine the minimum number of lines needed to cross out all zeros.

                          Task
  Assignee| A      B      C      D      E      |
__________|____________________________________|________
          |        |             |             |
     1    | 4      0      8      0      3      |
          |        |             |             |
     2    |-0------981------5------4------3------|
          |        |             |             |
     3    | 6      0      4      8      1      |


          |        |             |             |
     4    | 2      8      981    0      7      |
          |        |             |             |
     5    |-0------978------0------7------0------|
          |        |             |             |

Step :
Select the smallest number from all the uncovered numbers

                          Task
  Assignee| A      B      C      D      E      |
__________|____________________________________|________
     1    | 4      0      8      0      3      | 
     2    | 0      981    5      4      3      | 
     3    | 6      0      4      8      [1]    | 
     4    | 2      8      981    0      7      | 
     5    | 0      978    0      7      0      | 
          |                                    | 

Step :
The smallest number selected at the preceding step will be subtracted from every uncovered by 
number and added to every number at the intersection of covering lines automatically. 

                          Task
  Assignee| A      B      C      D      E      |
__________|____________________________________|________
     1    | 3      0      7      0      2      | 
     2    | 0      982    5      5      3      | 
     3    | 5      0      3      8      0      | 
     4    | 1      8      980    0      6      | 
     5    | 0      979    0      8      0      | 
          |                                    | 

Step :
Determine the minimum number of lines needed to cross out all zeros.

                          Task
  Assignee| A      B      C      D      E      |
__________|____________________________________|________
          |        |             |             |
     1    | 3      0      7      0      2      |
          |        |             |             |
     2    |-0------982------5------5------3------|
          |        |             |             |
     3    |-5------0------3------8------0------|
          |        |             |             |
     4    | 1      8      980    0      6      |
          |        |             |             |
     5    |-0------979------0------8------0------|
          |        |             |             |

Make the assignments.

                          Task
  Assignee| A      B      C      D      E      |
__________|____________________________________|________



     1    | 3      0      7      0      2      | 
     2    | [0]    982    5      5      3      | 
     3    | 5      0      3      8      0      | 
     4    | 1      8      980    [0]    6      | 
     5    | 0      979    0      8      0      | 

Job B is assigned to Worker 1
Job A is assigned to Worker 2
Job E is assigned to Worker 3
Job D is assigned to Worker 4
Job C is assigned to Worker 5

Total cost = 18 + 18 + 0 + 14 + 25 = 75
\end{lstlisting}
\end{solution}

\end{document}