\documentclass[titlepage, letterpaper]{article}
\usepackage[utf8]{inputenc}
\usepackage{fancyhdr}
\usepackage{amsmath}
\usepackage{extramarks}
\usepackage{enumitem}
\usepackage{amssymb}
\usepackage{booktabs}
\usepackage{tcolorbox}
\usepackage{gensymb}
\usepackage{booktabs}
\usepackage{graphicx}
\usepackage{caption}
\usepackage{hyperref}
\usepackage{listings}


\topmargin=-0.45in
\evensidemargin=0in
\oddsidemargin=0in
\textwidth=6.5in
\textheight=9.0in
\headsep=0.25in
\setlength{\parskip}{1ex}
\setlength{\parindent}{0ex}

\hypersetup{
    colorlinks=true
}

%
% You should change this things~
%

\newcommand{\mahclass}{Applied Mathematics}
\newcommand{\mahtitle}{Final Exam - Late Answers}
\newcommand{\mahdate}{\today}
\newcommand{\spacepls}{\vspace{5mm}}

%
% Header markings
%

\pagestyle{fancy}
\lhead{Final Exam}
\chead{}
\rhead{1170065 - Xavier Sánchez}
\lfoot{}
\rfoot{}


\renewcommand\headrulewidth{0.4pt}
\renewcommand\footrulewidth{0.4pt}
% \renewcommand{\familydefault}{\sfdefault} %The sans-serif font and the like
\newcommand{\qed}{\,\,\square}

% Alias for the Solution section header
\newcommand{\solution}{\textbf{\large Solución}}

%Alias for the new step section
\newcommand{\steppy}[1]{\textbf{\large #1}}

%
% My actual info
%

\title{
\vspace{1in}
\textbf{Tecnológico de Monterrey} \\
\vspace{0.5in}
\textmd{\mahclass} \\
\vspace{0.5in}
\textsc{\mahtitle}
\author{01170065  - MIT \\
Xavier Fernando Cuauhtémoc Sánchez Díaz \\
\texttt{xavier.sanchezdz@gmail.com}}
\date{\mahdate}
}

\begin{document}

\begin{titlepage}
    \maketitle
\end{titlepage}

%
% Actual document starts here~
%

\section{Exercise III.1 - LATE} % (fold)
\label{sec:exercise_iii_1}
We have a mean of 5 weeks and a standard deviation of 1.5 weeks.

We need to approximate that the probability of 13 or more batteries will be needed in 52 weeks, so we need $\dfrac{52}{13} = 4$ as a mean.

Therefore, using the formula, we have:
$$\dfrac{\sqrt{n} \cdot \overline{x} - \mu}{\sigma} = \dfrac{\sqrt{13} 4 - 1}{1.5} = -2.403700$$

Using Stattrek, it yields that $P(X \leq -2.4037) = 0.00812 \qed$

% section exercise_iii_1 (end)

\section{Exercise IV.2 LATE} % (fold)
\label{sec:exercise_iv_2}

% section exercise_iv_2 (end)

We have a variance of $\sigma^2$ which is normally distributed.
The thermostat was tested five times.
Therefore, we have 4 degrees of freedom.

Using Stattrek, we can obtain that $P(S^2/\sigma^2 \leq 1.8) = 0.23$.

Now $P(\chi \leq 1.15) = 0.11$, and $P(\chi \leq 0.85) = 0.07$, so the interval is $P(0.85 \leq \chi \leq 0.11) = 0.04\qed$.

\section{Exercise VI.1 - LATE} % (fold)
\label{sec:exercise_vi_1}

We have a standard deviation of 20.
Our null hypothesis is $H_0 \colon \mu_0 = 50$ versus $H_1 \colon \mu_0 \not = 50$.
Our sample average is 64.

The T-statistic can be calculated as:

$$\frac{\sqrt{n}(\overline{X} - \mu_0)}{\sigma}$$

Substituting accordingly, we have that $\dfrac{\sqrt{64}(52.5 - 50)}{20} = 1$.

The $p$-value is obtained via Stattrek, and yields $0.84134$.
Since it's a two sided test, then we need to calculate as $2P(Z \geq |z|) = 1 - 0.84134 = 0.15866 * 2 = 0.31732 \qed$.

For b), we calculate everything again: $\dfrac{\sqrt{64}(55 - 50)}{20} = 2$.
The $p$-value is $2(1 - 0.97725) = 0.0455 \qed$

For c), we calculate everything again: $\dfrac{\sqrt{64}(57.5 - 50)}{20} = 3$.
The $p$-value is $2(1 - 0.99865) = 0.0027 \qed$

% section exercise_vi_1 (end)

\section{Exercise VI.2 - LATE} % (fold)
\label{sec:exercise_vi_2}

We have a standard deviation of 1.2 pounds.
The hatchery claim is that $\mu_0 \geq 7.6$.

16 fish were sampled, with an average sample of 7.2 pounds.

The T-statistic is $\sqrt{16} \dfrac{7.2 - 7.6}{1.2} = -1.333$.

Therefore, the $p$-value is 0.09127, so 0.05 and 0.01 levels of confidence are ok.

We do not have enough statistical evidence to reject the null hypothesis, neither at 5\% or 1\% significance levels.

\section{Exercise IX.2 - LATE} % (fold)
\label{sec:exercise_ix_2}

% section exercise_ix_2 (end)

\section{Exercise X.2 - LATE} % (fold)
\label{sec:exercise_x_2}

Using IORTutorial we can get the following solution for the MachineCo job assignment problem:

\begin{lstlisting}[basicstyle=\tiny]
Solve an Assignment Problem Interactively:
Number of tasks:    4

Step :
Subtract the smallest number in each row from every number in the row.
Enter the results in a new table.

                       Task
  Assignee| A      B      C      D      | Row Min
__________|_____________________________|________
     1    | 999    5      8      7      | 5
     2    | 2      12     999    5      | 2
     3    | 7      999    3      9      | 3
     4    | 2      4      6      999    | 2
          |                             | 

Step :
Subtract the smallest number in each column of the new table from every number in the column.

                       Task
  Assignee| A      B      C      D      |
__________|_____________________________|________
     1    | 994    0      3      2      | 
     2    | 0      10     997    3      | 
     3    | 4      996    0      6      | 
     4    | 0      2      4      997    | 
Col Min   | 0      0      0      2      | 

Step :
Enter the results in a new table.


                       Task
  Assignee| A      B      C      D      |
__________|_____________________________|________
     1    | 994    0      3      0      | 
     2    | 0      10     997    1      | 
     3    | 4      996    0      4      | 
     4    | 0      2      4      995    | 
          |                             | 

Step :
Determine the minimum number of lines needed to cross out all zeros.

                       Task
  Assignee| A      B      C      D      |
__________|_____________________________|________
          | |             |             |
     1    |-994------0------3------0------|
          | |             |             |
     2    | 0      10     997    1      |
          | |             |             |
     3    | 4      996    0      4      |
          | |             |             |
     4    | 0      2      4      995    |
          | |             |             |



Step :
Select the smallest number from all the uncovered numbers

                       Task
  Assignee| A      B      C      D      |
__________|_____________________________|________
     1    | 994    0      3      0      | 
     2    | 0      10     997    [1]    | 
     3    | 4      996    0      4      | 
     4    | 0      2      4      995    | 
          |                             | 

Step :
The smallest number selected at the preceding step will be subtracted from every uncovered by 
number and added to every number at the intersection of covering lines automatically. 

                       Task
  Assignee| A      B      C      D      |
__________|_____________________________|________
     1    | 995    0      4      0      | 
     2    | 0      9      997    0      | 
     3    | 4      995    0      3      | 
     4    | 0      1      4      994    | 
          |                             | 

Step :
Determine the minimum number of lines needed to cross out all zeros.

                       Task
  Assignee| A      B      C      D      |
__________|_____________________________|________
          | |                    |      |
     1    |-995------0------4------0------|
          | |                    |      |
     2    | 0      9      997    0      |
          | |                    |      |
     3    |-4------995------0------3------|
          | |                    |      |
     4    | 0      1      4      994    |
          | |                    |      |

Make the assignments.

                       Task
  Assignee| A      B      C      D      |
__________|_____________________________|________
     1    | 995    0      4      0      | 
     2    | 0      9      997    0      | 
     3    | 4      995    0      3      | 
     4    | [0]    1      4      994    | 

Task B is assigned to Assignee 1
Task D is assigned to Assignee 2
Task C is assigned to Assignee 3
Task A is assigned to Assignee 4

Machine 1 will do Job 2
Machine 2 will do Job 4
Machine 3 will do Job 3
Machine 4 will do Job 1

Therefore, the optimal production cost is
5 + 5 + 3 + 2 = 15
\end{lstlisting}

% section exercise_x_2 (end)

\section{Exercise XI.2 - LATE} % (fold)
\label{sec:exercise_xi_2}

To solve this problem, we first need to solve the dual:

\begin{align*}
    W & = 6y_1 + 9y_2 \to \min \\
    \text{s.t.} \\
    2y_1 & + y_2 \leq 1 \\
    -y_1 & + 2y_2 = -2 \\
    -3y_1 & -2y_2 \geq 3 \\
    4y_1 & + 2y_2 \leq -4 \\
    \text{and} \\
    & y_1 \text{free}, y_2 \geq 0
\end{align*}

Nevertheless, the software mentions it is an unfeasible problem.

If we go back to the original problem, we can see that the value $X= (0,0,-2,0)$ satisfies all constraints.

Therefore, the original problem is unbounded $\qed$.

% section exercise_xi_2 (end)

\end{document}