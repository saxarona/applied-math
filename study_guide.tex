\documentclass[titlepage, letterpaper]{article}
\usepackage[utf8]{inputenc}
\usepackage{fancyhdr}
\usepackage{amsmath}
\usepackage{extramarks}
\usepackage{enumitem}
\usepackage{amssymb}
\usepackage{booktabs}
\usepackage{tcolorbox}
\usepackage{gensymb}
\usepackage{booktabs}
\usepackage{graphicx}
\usepackage{caption}
\usepackage{hyperref}


\topmargin=-0.45in
\evensidemargin=0in
\oddsidemargin=0in
\textwidth=6.5in
\textheight=9.0in
\headsep=0.25in
\setlength{\parskip}{1ex}
\setlength{\parindent}{0ex}

\hypersetup{
    colorlinks=true
}

%
% You should change this things~
%

\newcommand{\mahclass}{AppMath}
\newcommand{\mahtitle}{Study Guide}
\newcommand{\mahdate}{\today}
\newcommand{\spacepls}{\vspace{5mm}}

%
% Header markings
%

\pagestyle{fancy}
\lhead{Study Guide}
\chead{}
\rhead{}
\lfoot{}
\rfoot{}


\renewcommand\headrulewidth{0.4pt}
\renewcommand\footrulewidth{0.4pt}
\renewcommand{\familydefault}{\sfdefault} %The sans-serif font and the like
\newcommand{\qed}{\,\,\square}

% Alias for the Solution section header
\newcommand{\solution}{\textbf{\large Solución}}

%Alias for the new step section
\newcommand{\steppy}[1]{\textbf{\large #1}}

%
% My actual info
%

\title{
\vspace{1in}
\textbf{Tecnológico de Monterrey} \\
\vspace{0.5in}
\textmd{\mahclass} \\
\vspace{0.5in}
\textsc{\mahtitle}
\author{01170065  - MIT \\
Xavier Fernando Cuauhtémoc Sánchez Díaz \\
\texttt{xavier.sanchezdz@gmail.com}}
\date{\mahdate}
}

\begin{document}

\begin{titlepage}
    \maketitle
\end{titlepage}

%
% Actual document starts here~
%

\section{Ex. I.1} % (fold)
\label{sec:exercise_i_1}

The solution of EXI.1 is the Expected value of the dice being thrown once.
That is:
$$E[X] = \frac{1}{6} \times (1+2+3+4+5+6) = \frac{21}{6} = 3.5$$

% section exercise_i_1 (end)

\section{Ex. I.2} % (fold)
\label{sec:ex_i_2}

The expected amount of money drawn on the first draw would be $E[X] = \dfrac{1}{6} * 2 \times 5+10+25 = \dfrac{1}{2} \times 80 = \dfrac{40}{3}$.

The expected value is obtained per coin, so a second draw would yield the same expectation, $\dfrac{40}{3}$.

If we were to draw three coins, then the expected value would be 3 times that, which is 40.
Of course, the expected value does not change if we take the three coins altogether.

% section ex_i_2 (end)

\section{Ex. II.1} % (fold)
\label{sec:ex_ii_1}

To obtain the value of $\lambda$, we need to integrate from 0 to infinity, and equal to 1 (because it is a probability distribution), i.e.:

\begin{align}
    \int\limits_0^\infty \lambda e^{-x/100} & = \frac{\lambda e^{-x/100}}{\tfrac{-1}{100}} = 1 \\
    & = -100\lambda e^{-x/100}\Big|_0^\infty \\[3ex]
    & = -100\lambda e^{-\infty} - 100\lambda e^0 \\
    & = -100\lambda (0 - 1) = 1 \\
    & \therefore \lambda = \frac{1}{100} = 0.01 \qed
\end{align}

Now, we evaluate from 50 to 150, that is
$$-100 * 0.01(e^{-150/100} - e^{-50/100}) = -e^{-1.5} + e^{-0.5} \approx 0.3834$$

The probability that it will function less than 100 hours will be obtained by evaluation from 0 to 100:
$$-100 * 0.01 (e^{-100/100} - e^{-0/100}) = -e^{-1} + 1 \approx 0.6321$$
% section ex_ii_1 (end)

\section{Ex II.2} % (fold)
\label{sec:ex_ii_2}
This exercise is pretty much the same than the last one.
First we integrate with respect to $y$:

\begin{align}
\int\limits_0^\infty xe^{-x+y}dy & = \\
& = -xe^{-x+y} \Big|_0^\infty \\[3ex]
& = (-xe^{-x+\infty}) - (-xe^{-x + 0}) \\
& = (0) + xe^{-x} \qed 
\end{align}

For the second question, we need to integrate from 0 to infinity, but using integration by parts.
This time we integrate with respect to $x$:

Let $u = x$ and $dv = e^{-(x+y)}$, so that $du = dx$ and $v = - e^{-(x+y)}$.

\begin{align}
\int\limits_0^\infty xe^{-(x+y)}dx & = \\
& = -xe^{-(x+y)} + \int\limits_0^\infty e^{-(x+y)}\cdot (-dx) \\
& = -xe^{-(x+y)} - e^{-(x+y)}\Big|_0^\infty \\[3ex]
& = (-\infty e^{-\infty} - e^{-\infty}) - (0 - e^{-y}) \\
& = (0 - 0) - (0 - e^{-y}) \\
& = e^{-y} \qed
\end{align}

To know if both variables are independent, we can try multiplying both and see if the product is equal to the density:

$$xe^{-x} \times e^{-y} = xe^{-x-y} = xe^{-(x+y)}$$

So these two variables are, indeed, independent.
% section ex_ii_2 (end)

\section{Ex. III.1} % (fold)
\label{sec:ex_iii_1}

Question a) asks for $P(Z < 104)$, which can be calculated with the following formula:
\begin{equation}
    \frac{Z_v - \overline{x}}{s/\sqrt{n}}
\end{equation}

Substituting accordingly, we get $\dfrac{104 - 100}{20 / \sqrt{16}} = 0.8$.

Using the \href{http://stattrek.com/online-calculator/normal.aspx}{Stattrek website} (under normal distribution), the result yields that $P(Z < 0.8) = 0.78814 \qed$.

The second question asks us to find between two values, so we can use the same formula applied to 98, and then subtract it from the previous answer ($X < 104$):

$$\frac{98 - 100}{20/\sqrt{16} = -0.4}$$

Therefore, $P(X < -0.4) = 0.34458$, and so $P(-0.4 < X < 0.8) = 0.78814 - 0.34458 = 0.44356 \qed$
% section ex_iii_1 (end)

\section{Ex. III.2} % (fold)
\label{sec:ex_iii_2}
We can compute the exact value using the \href{http://stattrek.com/online-calculator/binomial.aspx}{Stattrek Binomial} calculator.

This value is exactly 0.0574595.

Now, the approximation needs some additional data.

First we need to calculate the expected value for this binomial problem:

\begin{equation}
E[X] = n \times p    
\end{equation}

So the expected value is $E[X] = 90$.
Now the Standard deviation can be calculated as:

\begin{equation}
    s = \sqrt{n \cdot p \cdot (1-p)}
\end{equation}

Therefore, $s = \sqrt{150 \times 0.6 \times 0.4} = 6$.

Now, we can approximate the value of $P(X \leq 80)$ by using the following equation:
\begin{equation}
    P(X\leq B_v) = \dfrac{B_v - E[X]}{s}
\end{equation}

Substituting accordingly, we have $\dfrac{80 - 90}{6} = -1.6667$. So now we can check in Stattrek, which yields 0.04779.

If we were to use the continuity correction, then $\dfrac{80.5 - 90}{6} = -1.58333$

Using Stattrek we get 0.05667, which is way closer to the exact value.
% section ex_iii_2 (end)

\section{Ex. IV.1} % (fold)
\label{sec:ex_iv_1}

Again, normal distribution with $\mu = 40$ and $\sigma = 4$.
The probability of the 2 in the next 4 years wil exceed 50 inches is calculated as follows:

$$P(X \geq 50) = P\left(\dfrac{50-40}{4}\right) = P(X \geq 2.5) = 1 - P(X \leq 2.5) = 0.00621 \qed$$.

To answer 2 out of 4 years, then we need to calculate it using the following equation:

\begin{equation}
    B_p(4,2) = \binom{4}{2} \times p^2 \times (1-p)^2
\end{equation}

Therefore, substituting accordingly, we have $B_p(4,2) = \binom{4}{2} \times 0.00621^2 \times (0.99379)^2 = 0.00022851 \qed$
% section ex_iv_1 (end)

\section{IV.2} % (fold)
\label{sec:iv_2}

Since $\chi$ is a chi-square random variable with 6 degrees of freedom, then we can use \href{http://stattrek.com/online-calculator/chi-square.aspx}{Stattrek website} to calculate it.
It yields $P(X \leq 6) = 0.58$.

For the second part, $P(3 \leq X \leq 9)$ we can calculate $P(X \leq 9) - P(X \leq 3) = 0.83 - 0.19 = 0.64\qed$

% section iv_2 (end)

\section{Ex. V.1} % (fold)
\label{sec:ex_v_1}

We have a standard deviation of 0.3.
We are calculating a sample needed, and since we're using 95\% confidence, the equation needed is:

\begin{equation}
    \mu \in \left(\overline{x} - 1.96\frac{\sigma}{\sqrt{n}}, \overline{x} + 1.96\frac{\sigma}{\sqrt{n}} \right)
\end{equation}

Multiplying, we can get that $\dfrac{0.588}{\sqrt{n}} \leq 0.1$.
Applying some algebra, we can get that $\sqrt{n} \geq 5.88$ and $n \geq 34.57$.
This means that, for this to be true, we need at least an $n=35$.

% section ex_v_1 (end)

\section{Ex. V.2} % (fold)
\label{sec:ex_v_2}

For this we need to determine a 95\% confidence interval for a lot of values.
We need to calculate the value of the mean and the standard deviation, all of them are for the sample (you can use excel or your calculator).

\(\overline{x} = 69.266667, s = 15.167948\)

After the input of these two values, we can ask the \href{https://www.mccallum-layton.co.uk/tools/statistic-calculators/confidence-interval-for-mean-calculator/#confidence-interval-for-mean-calculator}{software} and it yields an interval of $\pm$ 7.68, that is from 61.59 to 76.94.

Now, for the lower bound we will need to use this \href{https://home.ubalt.edu/ntsbarsh/Business-stat/otherapplets/Esteem.htm}{other page}.

The lower bound, as expressed for this page, is 60.86696.
% section ex_v_2 (end)

\section{Ex. VI.6} % (fold)
\label{sec:ex_vi_6}

We have an average of 1.6 mg per cigarette.
We also have a sample of 20.
The standard deviation of a cigarette content is 0.8 mg.
The average content of 20 cigarettes is 1.54.

Our null hypothesis $H_0 \colon \mu \geq 1.6$ versus $H_1 \colon \mu < 1.6$.

For that, we need the value of the test statistic, which is calculated as follows:

\begin{equation}
    \frac{\sqrt{n} (\overline{X} - \mu_0)}{\sigma}
\end{equation}

Substituting, we have that $\dfrac{\sqrt{20}(1.54 - 1.6)}{0.8} = -0.3354$.

To calculate the p-value for  this hypothesis, we have that $P(Z < -0.3354) = 0.36866$ which is strictly greater than the 0.05 level of significance, then we can't reject the null hypothesis.
That means that we don't have enough evidence to support the claim of the producer.

% section ex_vi_6 (end)

\section{Ex. VI.2} % (fold)
\label{sec:ex_vi_2}

We have a mean water use of 350 gallons per day.
20 randomly selected homes were investigated.

Using this \href{https://home.ubalt.edu/ntsbarsh/Business-stat/otherapplets/MeanTest.htm}{piece of software}, we can see that the T-statistic is 0.77784.

Using the \href{http://www.sjsu.edu/faculty/gerstman/StatPrimer/t-table.pdf}{table} we can see that $t_{0.05,19} = 1.729$, which is greater than 0.05, so we can accept at the 10\% level of significance.

The p-value, as wel as being given by the software as 0.22337, it needs to be multiplied by 2 to ensure it is inside the two-tails region, and therefore it is 0.4462.

If $\alpha$ were greater than the p-value, then the null hypothesis would be accepted at any significance level which is reasonable.

We could have also used the following equation:

\begin{equation}
    T = \frac{\sqrt{n}(\overline{X} - \mu_0)}{S}
\end{equation}
% section ex_vi_2 (end)

\section{Ex VII.1} % (fold)
\label{sec:ex_vii_1}

This is regression, where you will need \href{https://home.ubalt.edu/ntsbarsh/Business-stat/otherapplets/Regression.htm}{this page}.
Input some data, and compute.
Remember the intercept is Y and the slope is X.

Therefore, you'll find via software that $Y = -2.5104 +0.3232X$.
% section ex_vii_1 (end)

\section{VIII.1} % (fold)
\label{sec:viii_1}
For the ANOVA, you'll probably need \href{https://home.ubalt.edu/ntsbarsh/Business-stat/otherapplets/ANOVA.htm}{this page}.
You need to check if the p-value is greater than your confidence level (0.05), then you can be certain about the difference between means.
% section viii_1 (end)

\section{Ex. IX.1} % (fold)
\label{sec:ex_ix_1}

Let's try to do Big M!



% section ex_ix_1 (end)
\end{document}