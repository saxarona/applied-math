\documentclass[titlepage, letterpaper, fleqn]{article}
\usepackage[utf8]{inputenc}
\usepackage{fancyhdr} % fancy headers, of course!
\usepackage{amsmath} % what do you think?
\usepackage{amsthm} % theorems!
\usepackage{extramarks} % more cute things
\usepackage{enumitem} % i'm not sure...
\usepackage{multicol} % multicolumn...?
\usepackage{amssymb} % more symbols
\usepackage{booktabs} % cool looking tables
\usepackage{tikz} %venn and shizzle

\topmargin=-0.45in
\evensidemargin=0in
\oddsidemargin=0in
\textwidth=6.5in
\textheight=9.0in
\headsep=0.25in


%
% You should change this things~
%

\newcommand{\mahteacher}{Dr. Viacheslav Kalashnikov}
\newcommand{\mahclass}{Applied Mathematics}
\newcommand{\mahtitle}{Activity 6 - Exercise 1.2.3: Reflexivity and Transitivity}
\newcommand{\mahdate}{August 31, 2016}
\newcommand{\spacepls}{\vspace{5mm}}
\renewcommand\qedsymbol{\(\blacksquare\)}

%
% Header markings
%

\pagestyle{fancy}
\lhead{1170065 - Xavier Sánchez}
\chead{}
\rhead{}
\lfoot{}
\rfoot{}


\renewcommand\headrulewidth{0.4pt}
\renewcommand\footrulewidth{0.4pt}

\setlength\parindent{0pt}


%
% Create Problem Sections (stolen directly from jdavis/latex-homework-template @ github!)
%

\newcommand{\enterProblemHeader}[1]{
\nobreak\extramarks{}{Problem \arabic{#1} continued on next page\ldots}\nobreak{}
\nobreak\extramarks{Problem \arabic{#1} (continued)}{Problem \arabic{#1} continued on next page\ldots}\nobreak{}
}

\newcommand{\exitProblemHeader}[1]{
\nobreak\extramarks{Problem \arabic{#1} (continued)}{Problem \arabic{#1} continued on next page\ldots}\nobreak{}
\stepcounter{#1}
\nobreak\extramarks{Problem \arabic{#1}}{}\nobreak{}
}

\setcounter{secnumdepth}{0}
\newcounter{partCounter}
\newcounter{homeworkProblemCounter}
\setcounter{homeworkProblemCounter}{1}
\nobreak\extramarks{Exercise \arabic{homeworkProblemCounter}}{}\nobreak{}

% Alias for the Solution section header
\newcommand{\solution}{\textbf{\Large Solution}}

%Alias for the new step section
\newcommand{\steppy}[1]{\textbf{\large #1}}

%
% Homework Problem Environment
%
% This environment takes an optional argument. When given, it will adjust the
% problem counter. This is useful for when the problems given for your
% assignment aren't sequential. See the last 3 problems of this template for an
% example.
%
\newenvironment{homeworkProblem}[1][-1]{
\ifnum#1>0
\setcounter{homeworkProblemCounter}{#1}
\fi
\section{Exercise \arabic{homeworkProblemCounter}}
\setcounter{partCounter}{1}
\enterProblemHeader{homeworkProblemCounter}
}{
\exitProblemHeader{homeworkProblemCounter}
}

%
% Venn diagrams defs
%

% \def\firstcircle{(0,0) circle (1.5cm)}
% \def\secondcircle{(0:2cm) circle (1.5cm)}
% \colorlet{circle edge}{blue!50}
% \colorlet{circle area}{blue!20}

% \tikzset{filled/.style={fill=circle area, draw=circle edge, thick},
%     outline/.style={draw=circle edge, thick}}

%
% My actual info
%

\title{
\vspace{1in}
\textbf{Tecnológico de Monterrey} \\
\vspace{0.5in}
\textmd{\mahclass} \\
\large{\textit{\mahteacher}} \\
\vspace{0.5in}
\textsc{\mahtitle}
\author{01170065  - MIT \\
Xavier Fernando Cuauhtémoc Sánchez Díaz \\
\texttt{mail@gmail.com}}
\date{\mahdate}
}

\begin{document}

\begin{titlepage}
\maketitle
\end{titlepage}

%
% Actual document starts here~
%

{\large \textbf{a)} Show that \(R\) is reflexive over \(A\) iff \(I_A \subseteq R\).
\(I_A\) is the identity relation over \(A\).}

\begin{proof}
Assume \(R\) is reflexive over \(A\).\\
Therefore, \((a,a) \in R : a \in A\), and so \(a,a \in I_A\).\\
Because \((a,a) \in R\) and \((a,a) \in I_A\), then \(I_A \subseteq R\).

\spacepls

Now assume \(I_A \subseteq R\).\\
Therefore, \((a,a) \in I_A : a \in A\).\\
And if \(I_A \subseteq R\), then \((a,a) \in R : a \in A\).\\
Therefore, \(R\) is reflexive over \(A\).
\end{proof}

\spacepls

{\large \textbf{b)} Show that the converse of a relation \(R\) that is reflexive over \(A\) is also reflexive over \(A\).}

\begin{proof}
Assume \(R\) is reflexive over A.\\
So by definition, \((a,a) \in R : a \in A\).\\
However, since domain and range are the same, then \((a,a) \in R^{-1} : a \in A\).\\
Hence, \(R^{-1}\) is also reflexive over A.

\spacepls

Now assume \(R^{-1}\) is reflexive over A.\\
So by definition, \((a,a) \in R^{-1} : a \in A\).\\
However, since domain and range are the same, then \((a,a) \in R : a \in A\).\\
Therefore, \(R\) is also reflexive over A.
\end{proof}

\spacepls

{\large \textbf{c)} Show that \(R\) is transitive iff \(R \circ R \subseteq R\)}.

\begin{proof}
Let \(a \in A, b \in B, c \in C\).\\
Let \((a,b) \in R, (b,c) \in R\) and \((a,c) \in R\).\\
By definition, \((a,c) \in R \circ R\),\\
and since \((a,c) \in R\), then \(R \circ R \subseteq R\).\\
Hence, \(R\) is transitive.

\spacepls

Now assume \(R\) is transitive.\\
If \(R\) is transitive, then \((a,b) \in R, (b,c) \in R\) and also \((a,c) \in R\),\\
and by definition, \((a,c) \in R \circ R\).\\
And since \((a,c) \in R\) and \((a,c) \in R \circ R\), then \(R \circ R \subseteq R\).
\end{proof}

\end{document}