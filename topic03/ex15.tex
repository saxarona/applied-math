\documentclass[titlepage, letterpaper, fleqn]{article}
\usepackage[utf8]{inputenc}
\usepackage{fancyhdr} % fancy headers, of course!
\usepackage{amsmath} % what do you think?
\usepackage{amsthm} % theorems!
\usepackage{extramarks} % more cute things
\usepackage{enumitem} % i'm not sure...
\usepackage{multicol} % multicolumn...?
\usepackage{amssymb} % more symbols
\usepackage{MnSymbol} % moar symbols?
\usepackage{booktabs} % cool looking tables
\usepackage{tikz} %venn and shizzle
\usepackage{tikz-qtree-compat} %tableaux
\usepackage{lipsum} %lorem ipsum dolor sit amet f u
\usepackage{mathrsfs} %math script for calligraphic scripting, I GUESS

\topmargin=-0.45in
\evensidemargin=0in
\oddsidemargin=0in
\textwidth=6.5in
\textheight=9.0in
\headsep=0.25in


%
% You should change this things~
%

\newcommand{\mahteacher}{Dr. Viacheslav Kalashnikov}
\newcommand{\mahclass}{Applied Mathematics}
\newcommand{\mahtitle}{Topic III - Activity 15}
\newcommand{\mahdate}{November 09, 2016}
\newcommand{\spacepls}{\vspace{5mm}}
\newcommand{\until}{\mathscr{U}}
\renewcommand\qedsymbol{\(\blacksquare\)}

%
% Header markings
%

\pagestyle{fancy}
\lhead{1170065 - Xavier Sánchez}
\chead{}
\rhead{}
\lfoot{}
\rfoot{}


\renewcommand\headrulewidth{0.4pt}
\renewcommand\footrulewidth{0.4pt}

\setlength\parindent{0pt}


%
% Create Problem Sections (stolen directly from jdavis/latex-homework-template @ github!)
%

\newcommand{\enterProblemHeader}[1]{
\nobreak\extramarks{}{Problem \arabic{#1} continued on next page\ldots}\nobreak{}
\nobreak\extramarks{Problem \arabic{#1} (continued)}{Problem \arabic{#1} continued on next page\ldots}\nobreak{}
}

\newcommand{\exitProblemHeader}[1]{
\nobreak\extramarks{Problem \arabic{#1} (continued)}{Problem \arabic{#1} continued on next page\ldots}\nobreak{}
\stepcounter{#1}
\nobreak\extramarks{Problem \arabic{#1}}{}\nobreak{}
}

\setcounter{secnumdepth}{0}
\newcounter{partCounter}
\newcounter{homeworkProblemCounter}
\setcounter{homeworkProblemCounter}{1}
\nobreak\extramarks{Exercise \arabic{homeworkProblemCounter}}{}\nobreak{}

%Solution Environment
\newenvironment{solution}
{\renewcommand\qedsymbol{$\square$}\begin{proof}[Solution]}
{\end{proof}}

% Alias for the Solution section header
%\newcommand{\solution}{\textbf{\Large Solution}}

%Alias for the new step section
\newcommand{\steppy}[1]{\textbf{\large #1}}

%
% Homework Problem Environment
%
% This environment takes an optional argument. When given, it will adjust the
% problem counter. This is useful for when the problems given for your
% assignment aren't sequential. See the last 3 problems of this template for an
% example.
%
\newenvironment{homeworkProblem}[1][-1]{
\ifnum#1>0
\setcounter{homeworkProblemCounter}{#1}
\fi
\section{Exercise \arabic{homeworkProblemCounter}}
\setcounter{partCounter}{1}
\enterProblemHeader{homeworkProblemCounter}
}{
\exitProblemHeader{homeworkProblemCounter}
}

%
% My actual info
%

\title{
\vspace{1in}
\textbf{Tecnológico de Monterrey} \\
\vspace{0.5in}
\textmd{\mahclass} \\
\large{\textit{\mahteacher}} \\
\vspace{0.5in}
\textsc{\mahtitle}\\
\textsc{3.5.1 Bayes' Theorem}\\
\textsc{3.5.2 Bayes' Theorem}\\
\textsc{3.5.3 Bayes' Theorem}\\
\author{01170065  - MIT \\
Xavier Fernando Cuauhtémoc Sánchez Díaz \\
\texttt{mail@gmail.com}}
\date{\mahdate}
}

\begin{document}

\begin{titlepage}
\maketitle
\end{titlepage}

%
% Actual document starts here~
%

\section{Exercise 3.5.1}

{\large One urn contains 12 blue balls and 7 white balls, and a second urn contains 8 blue balls and 19 white balls.
An urn is selected at random, and a ball is chosen from the urn.

\textbf{a)} What is the probability that the chosen ball is blue?}

\begin{solution}
First we have that \(P(B|U_1) = \frac{12}{19}, P(B|U_2) = \frac{8}{27}\).

Since both urns are equally likely to be chosen, then we have \(P(U_1) = P(U_2) = \frac{1}{2}\).

Then, by total probability, \(P(B) = P(B \cap U_1) + P(B \cap U_2)\).
To obtain the probability of choosing both a blue ball and a specific urn, conditional probability equation can be used:
\[P(B \cap U_1) = P(B|U_1) \cdot P(U_1) = \frac{12}{19} \cdot \frac{1}{2} = \frac{6}{19} \\
P(B \cap U_2) = P(B|U_2) \cdot P(U_2) = \frac{8}{27} \cdot \frac{1}{2} = \frac{4}{27}\]
\[P(B) = \frac{6}{19} + \frac{4}{27} = \frac{238}{513} \approx 0.4639 \qedhere \]
\end{solution}

\spacepls

{\large \textbf{b)} If the chosen ball is blue, what is the probability that it came from the first urn?}

\begin{solution}
\[P(U_1|B) = \frac{P(B|U_1) \cdot P(U_1)}{P(B)} = \frac{\frac{12}{19} \cdot \frac{1}{2}}{\frac{238}{513}} = \frac{81}{19} \approx 0.6806 \qedhere\]
\end{solution}

\section{Exercise 3.5.2}

{\large Redo the last exercise assuming that the first urn contains 4 blue balls and 16 white balls,
and the second urn contains 10 blue balls and 9 white balls.}

\begin{solution}
First we have that \(P(B|U_1) = \frac{1}{5}\) and \(P(B|U_2) = \frac{10}{19}\).

Again, both urns are equally likely to be chosen, hence \(P(U_1) = P(U_2) = \frac{1}{2}\).

The probability of withdrawing a blue ball from a random urn is obtained as follows:

\begin{align*}
P(B) & = P(B \cap U_1) + P(B \cap U_2)
\\ & = P(B|U_1) \cdot P(U_1) + P(B|U_2) \cdot P(U_2)
\\ & = \frac{1}{5} \cdot \frac{1}{2} + \frac{10}{19} \cdot \frac{1}{2}
\\ & = \frac{69}{190}
\\ & \approx 0.3631 \,\,\square
\end{align*}

Now for the probability that the blue ball came from the first urn:

\[P(U_1|B) = \frac{P(B|U_1) \cdot P(U_1)}{P(B)} = \frac{\frac{1}{5} \cdot \frac{1}{2}}{\frac{69}{190}} = \frac{19}{69} \approx 0.2753 \qedhere\]

\end{solution}

\spacepls

\section{Exercise 3.5.3}

{\large A drug-screening test is used in a large population of people of whom 4\% actually use drugs.
Suppose that the false positive rate is 3\% and the false negative rate is 2\% .
Thus a person who uses drugs tests positive for them 98\% of the time,
and a person who doesn't use drugs tests negative for them 97\% of the time.

\textbf{a)} What is the probability that a randomly chosen person who tests positive for drugs actually uses drugs?}

\begin{solution}
Let \(T\) be the event that the person tests positive and \(D_1\) the event that the person actually uses drugs.

We're looking for \(P(D|T)\), which can be calculated with the Bayes' Theorem.
We have the following data: \(P(T|D) = 0.98, P(\neg T | \neg D) = 0.97, P(D) = 0.04\).

So, the probability of a person using drugs that tested positive is calculated as follows:

\[P(D|T) = \frac{P(T|D) \cdot P(D)}{P(T)} = \frac{P(T|D) \cdot P(D)}{P(T|D) \cdot P(D) + P(T|\neg D) \cdot P(\neg D)}\]

The rest of the probabilities can be obtained as the complement of their counterparts: \(P(T|\neg D) = 0.03\) and \(P(\neg D) = 0.96\).

\begin{align*}
P(D|T) & =  \frac{0.98 \cdot 0.04}{0.98 \cdot 0.04 + 0.03 \cdot 0.96)}\\
& \approx 0.5764
\end{align*}

Thus the probability that a person who tested positive actually uses drugs is around 58\%.
\end{solution}

\spacepls

{\large \textbf{b)} What is the probability that a randomly chosen person who tests negative for drugs does not use drugs?}

\begin{solution}
The same approach is used for this problem.

\begin{align*}
P(\neg D | \neg T) & = \frac{P(\neg T | \neg D) \cdot P(\neg D)}{P(\neg T | \neg D) \cdot P(\neg D) + P(\neg T | D) \cdot P(D)} \\[2ex]
& = \frac{0.97 \cdot 0.96}{0.97 \cdot 0.96 + 0.02 \cdot 0.04}\\
& \approx 0.9991
\end{align*}
Thus the probability that a person who tested negative does not use drugs is around 99\%.
\end{solution}
\end{document}