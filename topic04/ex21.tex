\documentclass[titlepage, letterpaper, fleqn]{article}
\usepackage[utf8]{inputenc}
\usepackage{fancyhdr} % fancy headers, of course!
\usepackage{amsmath} % what do you think?
\usepackage{amsthm} % theorems!
\usepackage{extramarks} % more cute things
\usepackage{enumitem} % i'm not sure...
\usepackage{multicol} % multicolumn...?
\usepackage{amssymb} % more symbols
\usepackage{MnSymbol} % moar symbols?
\usepackage{booktabs} % cool looking tables
\usepackage{tikz} %venn and shizzle
\usepackage{tikz-qtree-compat} %tableaux
\usepackage{lipsum} %lorem ipsum dolor sit amet f u
\usepackage{mathrsfs} %math script for calligraphic scripting, I GUESS

\topmargin=-0.45in
\evensidemargin=0in
\oddsidemargin=0in
\textwidth=6.5in
\textheight=9.0in
\headsep=0.25in


%
% You should change this things~
%

\newcommand{\mahteacher}{Dr. Viacheslav Kalashnikov}
\newcommand{\mahclass}{Applied Mathematics}
\newcommand{\mahtitle}{Topic IV - Activity 21}
\newcommand{\mahdate}{November 16, 2016}
\newcommand{\spacepls}{\vspace{5mm}}
\newcommand{\until}{\mathscr{U}}
\renewcommand\qedsymbol{\(\blacksquare\)}

%
% Header markings
%

\pagestyle{fancy}
\lhead{1170065 - Xavier Sánchez}
\chead{}
\rhead{}
\lfoot{}
\rfoot{}


\renewcommand\headrulewidth{0.4pt}
\renewcommand\footrulewidth{0.4pt}

\setlength\parindent{0pt}


%
% Create Problem Sections (stolen directly from jdavis/latex-homework-template @ github!)
%

\newcommand{\enterProblemHeader}[1]{
\nobreak\extramarks{}{Problem \arabic{#1} continued on next page\ldots}\nobreak{}
\nobreak\extramarks{Problem \arabic{#1} (continued)}{Problem \arabic{#1} continued on next page\ldots}\nobreak{}
}

\newcommand{\exitProblemHeader}[1]{
\nobreak\extramarks{Problem \arabic{#1} (continued)}{Problem \arabic{#1} continued on next page\ldots}\nobreak{}
\stepcounter{#1}
\nobreak\extramarks{Problem \arabic{#1}}{}\nobreak{}
}

\setcounter{secnumdepth}{0}
\newcounter{partCounter}
\newcounter{homeworkProblemCounter}
\setcounter{homeworkProblemCounter}{1}
\nobreak\extramarks{Exercise \arabic{homeworkProblemCounter}}{}\nobreak{}

%Solution Environment
\newenvironment{solution}
{\renewcommand\qedsymbol{$\square$}\begin{proof}[Solution]}
{\end{proof}}

% Alias for the Solution section header
%\newcommand{\solution}{\textbf{\Large Solution}}

%Alias for the new step section
\newcommand{\steppy}[1]{\textbf{\large #1}}

%
% Homework Problem Environment
%
% This environment takes an optional argument. When given, it will adjust the
% problem counter. This is useful for when the problems given for your
% assignment aren't sequential. See the last 3 problems of this template for an
% example.
%
\newenvironment{homeworkProblem}[1][-1]{
\ifnum#1>0
\setcounter{homeworkProblemCounter}{#1}
\fi
\section{Exercise \arabic{homeworkProblemCounter}}
\setcounter{partCounter}{1}
\enterProblemHeader{homeworkProblemCounter}
}{
\exitProblemHeader{homeworkProblemCounter}
}

%
% My actual info
%

\title{
\vspace{1in}
\textbf{Tecnológico de Monterrey} \\
\vspace{0.5in}
\textmd{\mahclass} \\
\large{\textit{\mahteacher}} \\
\vspace{0.5in}
\textsc{\mahtitle}\\
\textsc{Hypothesis Testing}\\
\textsc{4.4.1}\\
\textsc{4.4.2}\\
\textsc{4.4.3}\\
\textsc{4.4.4}\\
\author{01170065  - MIT \\
Xavier Fernando Cuauhtémoc Sánchez Díaz \\
\texttt{mail@gmail.com}}
\date{\mahdate}
}

\begin{document}

\begin{titlepage}
\maketitle
\end{titlepage}

%
% Actual document starts here~
%

\section{Exercise 4.4.1}

{\large A population distribution is known to have a standard deviation 20.
Determine the p-value of a test of the hypothesis that the population mean is equal to 50, if the average of a sample of 64 observations is 52.5.}

\begin{solution}
Let $H_0$ be our hypothesis that $\mu = \mu_0$.
We will accept $H_0$ if $\frac{\sqrt{n}}{\sigma}|\overline{X}-\mu_0| \leq z_{\alpha/2}$.

Substituting the correspondent values, we have that the test statistic is $\frac{\sqrt{64}}{20}|52.5-50| = 1 \leq 1.96$

Now $P(|Z| > 1) = 0.159$.
\end{solution}

\spacepls

{\large \textbf{b)} Use now an average sample of 55.}

\begin{solution}
Let $H_0$ be our hypothesis that $\mu = \mu_0$.
The test statistic is calculated as follows: $\frac{\sqrt{64}}{20}|55-50| = 2$ which is greater than 1.96, so we reject $H_0$.

The p-value is $P(|Z| > 2) = 0.023$.
\end{solution}

\spacepls

{\large \textbf{c)} What if the average sample is 57.5?}

\begin{solution}
The test statistic is calculated as follows: $\frac{\sqrt{64}}{20}|57.5-50| = 3$, so we reject $H_0$.

The p-value is $P(|Z| > 3) = 0.001$ which is way too low for us to accept $H_0$.
\end{solution}

\spacepls

\section{Exercise 4.4.2}

{\large The weights of salmon grown at a commercial hatchery are normally distributed with a standard deviation of 1.2 pounds.
The hatchery claims that the mean weight of this year's crop is at least 7.6 pounds.
Suppose a random sample of 16 fish yielded an average weight of 7.2 pounds.
Is this strong enough evidence to reject the hatchery's claims at the 5\% level of significance?}

\begin{solution}
Let $H_0$ be the hypothesis that $\mu_0 = 7.6$.
The test statistic is $\frac{\sqrt{16}}{1.2}(7.2-7.6) = 1.33333$.
The p-value is $1 - \Phi(1.33333) = 0.09121$, so this claim would be accepted.
\end{solution}

\spacepls

{\large \textbf{b)} What about 1\% of significance?}

\begin{solution}
The test statistic would be calculated in the same manner: $\frac{\sqrt{16}}{1.2}(7.2-7.6) = 1.33333$.
The p-value is the same, 0.09121, and since it is greater than the 1\% level of significance, then it is also accepted.
\end{solution}

\spacepls

\section{Exercise 4.4.3}

{\large There is some variability in the amount of phenobarbitol in each capsule sold by a manufacturer.
However, the manufacturer claims that the mean value is 20 mg.
To test this, a sample of 25 pills yielded a sample mean of 19.7 with a sample standard deviation of 1.3.

What inference would you draw from these data?
In particular, are the data strong enough evidence to discredit the claim of the manufacturer?
Use the 5\% level of significance.}

\begin{solution}
Let $H_0$ be our hypothesis that $\mu = 20$mg.
The test statistic would be calculated as $\frac{\sqrt{25}}{1.3}|19.7-20| = 1.15384$.

The p-value is $0.12428$, so at 5\% level of significance this is accepted and we can't discredit the manufacturer.
\end{solution}

\spacepls

\section{Exercise 4.4.4}

{\large A pharmaceutical house produces a certain drug item whose weight has a standard deviation of 0.5 milligrams.
The company's research team has proposed a new method of producing the drug.
However, this entails some costs and will be adopted only if there is strong evidence that the standard deviation of the weight of the items will drop to below 0.4 milligrams.
If a sample of 10 items is produced and has the following weights, should the new method be adopted?

5.728  5.731  5.727  5.724  5.723,
5.718  5.726  5.722  5.719  5.722.}

\begin{solution}
Let $H_0$ our hypothesis that $\sigma \leq 0.4$.
The p-value of the test data is the probability that a chi-square random variable with 9 degrees of freedom would exceed the observed value $c$.

The test statistic $c$ is calculated as $\frac{(n-1)S^2}{\sigma^2_0} = \frac{9 \times 1.6444 \times 10^{-5}}{0.4^2} = 9.25 \times 10^{-4}$

It is important to note that $1.6444\times 10^{-5}$ has been estimated with the sample variance using the data provided.

The p-value is $P(\chi^2_{n-1} > c) = P(\chi^2_{9} > 9.25\times10^{-4}) \approx 1$.

Therefore, the hypothesis $H_0$ is rejected and the method is not adopted.
\end{solution}
\end{document}