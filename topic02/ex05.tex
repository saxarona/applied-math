\documentclass[titlepage, letterpaper, fleqn]{article}
\usepackage[utf8]{inputenc}
\usepackage{fancyhdr} % fancy headers, of course!
\usepackage{amsmath} % what do you think?
\usepackage{amsthm} % theorems!
\usepackage{extramarks} % more cute things
\usepackage{enumitem} % i'm not sure...
\usepackage{multicol} % multicolumn...?
\usepackage{amssymb} % more symbols
\usepackage{booktabs} % cool looking tables
\usepackage{tikz} %venn and shizzle
\usepackage{lipsum} %lorem ipsum dolor sit amet f u

\topmargin=-0.45in
\evensidemargin=0in
\oddsidemargin=0in
\textwidth=6.5in
\textheight=9.0in
\headsep=0.25in


%
% You should change this things~
%

\newcommand{\mahteacher}{Dr. Viacheslav Kalashnikov}
\newcommand{\mahclass}{Applied Mathematics}
\newcommand{\mahtitle}{Topic II - Activity 4}
\newcommand{\mahdate}{September 7, 2016}
\newcommand{\spacepls}{\vspace{5mm}}
\renewcommand\qedsymbol{\(\blacksquare\)}

%
% Header markings
%

\pagestyle{fancy}
\lhead{1170065 - Xavier Sánchez}
\chead{}
\rhead{}
\lfoot{}
\rfoot{}


\renewcommand\headrulewidth{0.4pt}
\renewcommand\footrulewidth{0.4pt}

\setlength\parindent{0pt}


%
% Create Problem Sections (stolen directly from jdavis/latex-homework-template @ github!)
%

\newcommand{\enterProblemHeader}[1]{
\nobreak\extramarks{}{Problem \arabic{#1} continued on next page\ldots}\nobreak{}
\nobreak\extramarks{Problem \arabic{#1} (continued)}{Problem \arabic{#1} continued on next page\ldots}\nobreak{}
}

\newcommand{\exitProblemHeader}[1]{
\nobreak\extramarks{Problem \arabic{#1} (continued)}{Problem \arabic{#1} continued on next page\ldots}\nobreak{}
\stepcounter{#1}
\nobreak\extramarks{Problem \arabic{#1}}{}\nobreak{}
}

\setcounter{secnumdepth}{0}
\newcounter{partCounter}
\newcounter{homeworkProblemCounter}
\setcounter{homeworkProblemCounter}{1}
\nobreak\extramarks{Exercise \arabic{homeworkProblemCounter}}{}\nobreak{}

% Alias for the Solution section header
\newcommand{\solution}{\textbf{\Large Solution}}

%Alias for the new step section
\newcommand{\steppy}[1]{\textbf{\large #1}}

%
% Homework Problem Environment
%
% This environment takes an optional argument. When given, it will adjust the
% problem counter. This is useful for when the problems given for your
% assignment aren't sequential. See the last 3 problems of this template for an
% example.
%
\newenvironment{homeworkProblem}[1][-1]{
\ifnum#1>0
\setcounter{homeworkProblemCounter}{#1}
\fi
\section{Exercise \arabic{homeworkProblemCounter}}
\setcounter{partCounter}{1}
\enterProblemHeader{homeworkProblemCounter}
}{
\exitProblemHeader{homeworkProblemCounter}
}

%
% Venn diagrams defs
%

% \def\firstcircle{(0,0) circle (1.5cm)}
% \def\secondcircle{(0:2cm) circle (1.5cm)}
% \colorlet{circle edge}{blue!50}
% \colorlet{circle area}{blue!20}

% \tikzset{filled/.style={fill=circle area, draw=circle edge, thick},
%     outline/.style={draw=circle edge, thick}}

%
% My actual info
%

\title{
\vspace{1in}
\textbf{Tecnológico de Monterrey} \\
\vspace{0.5in}
\textmd{\mahclass} \\
\large{\textit{\mahteacher}} \\
\vspace{0.5in}
\textsc{\mahtitle}\\
\textsc{2.1.1 Propositional Logic}\\
\author{01170065  - MIT \\
Xavier Fernando Cuauhtémoc Sánchez Díaz \\
\texttt{xavier.sanchezdz@gmail.com}}
\date{\mahdate}
}

\begin{document}

\begin{titlepage}
\maketitle
\end{titlepage}

%
% Actual document starts here~
%

\section{Exercise 2.1.2}

{\large \textbf{a)} Prove that \(A \wedge (B \vee C) \equiv (A \wedge B) \vee (A \wedge C)\):}

\begin{proof}
Conjunction is distributive over disjunction, so \(A \wedge (B \vee C) \equiv (A \wedge B) \vee (A \wedge C)\).

\begin{table}[h!]
\centering
\begin{tabular}{@{}ccccc@{}}
\toprule
\multicolumn{1}{l}{$A$} & \multicolumn{1}{l}{$B$} & \multicolumn{1}{l}{$C$} & \multicolumn{1}{l}{$A \wedge (B \vee C)$} & \multicolumn{1}{l}{$(A \wedge C) \vee (A \wedge C)$} \\ \midrule
T & T & T & T & T \\
T & T & F & T & T \\
T & F & T & T & T \\
T & F & F & F & F \\
F & T & T & F & F \\
F & T & F & F & F \\
F & F & T & F & F \\
F & F & F & F & F \\ \bottomrule
\end{tabular}
\caption{Truth table for \(A \wedge (B \vee C) \equiv (A \wedge B) \vee (A \wedge C)\)}
\label{fig:a}
\end{table}
\end{proof}

{\large \textbf{b)} Prove that \(A \vee B \equiv \neg(\neg A \wedge \neg B)\):}

\begin{proof}
This is one of de Morgan's laws, so \(A \vee B \equiv \neg(\neg A \wedge \neg B)\).

\begin{table}[h!]
\centering
\begin{tabular}{@{}cccc@{}}
\toprule
\multicolumn{1}{l}{$A$} & \multicolumn{1}{l}{$B$} & \multicolumn{1}{l}{$A \vee B$} & \multicolumn{1}{l}{$\neg (\neg A \wedge \neg B)$} \\ \midrule
T & T & T & T \\
T & F & T & T \\
F & T & T & T \\
F & F & F & F \\ \bottomrule
\end{tabular}
\caption{Truth table for \(A \vee B\)}
\label{fig:b}
\end{table}
\end{proof}

{\large \textbf{c)} Prove that \(A \wedge B \equiv \neg(\neg A \vee \neg B)\):}

\begin{proof}
This is another de Morgan's laws, so \(A \wedge B \equiv \neg(\neg A \vee \neg B)\).

\begin{table}[h!]
\centering
\begin{tabular}{@{}cccc@{}}
\toprule
\multicolumn{1}{l}{$A$} & \multicolumn{1}{l}{$B$} & \multicolumn{1}{l}{$A \wedge B$} & \multicolumn{1}{l}{$\neg (\neg A \vee \neg B)$} \\ \midrule
T & T & T & T \\
T & F & F & F \\
F & T & F & F \\
F & F & F & F \\ \bottomrule
\end{tabular}
\caption{Truth table for \(A \wedge B\)}
\label{fig:c}
\end{table}
\end{proof}

\pagebreak

{\large \textbf{d)} Prove that \(A \implies B \equiv \neg A 
\vee B\):}

\begin{proof}
By definition, implication is always true except when the hypothesis is true and the conclusion is false. Likewise, the only way to get a false in the right hand side of the identity, is for \(B\) to be false.

\begin{table}[h!]
\centering
\begin{tabular}{@{}cccc@{}}
\toprule
\multicolumn{1}{l}{$A$} & \multicolumn{1}{l}{$B$} & \multicolumn{1}{l}{$A \implies B$} & \multicolumn{1}{l}{$\neg A \vee B$} \\ \midrule
T & T & T & T \\
T & F & F & F \\
F & T & T & T \\
F & F & T & T \\ \bottomrule
\end{tabular}
\caption{Truth table for \(A \implies B\)}
\label{fig:d}
\end{table}
\end{proof}

{\large \textbf{e)} Prove that \(A \implies B \equiv \neg(A \wedge \neg B)\):}

\begin{proof}
By applying de Morgan's law, the RHS of the identity becomes \(\neg A 
\vee B\), and its truth table is identical to the one generated for \(A \implies B\) and which is shown in the previous exercise.
Therefore, \(A \implies B\)\ is equivalent to \(\neg(A \wedge \neg B)\).
\end{proof}

\spacepls

{\large \textbf{f)} Prove that \((A \oplus B) \oplus B \equiv A\):}

\begin{proof}
By definition, Exclusive OR will only output TRUE when both inputs are different.
Now, since Exclusive OR is associative, then \((A \oplus B) \oplus B \equiv A \oplus (B \oplus B)\).
And since \(B \oplus B\) will never output TRUE, then the value of \(A \oplus (B \oplus B)\) depends entirely on \(A\).
Therefore, \((A \oplus B) \oplus B\) is equivalent to \(A\).
\end{proof}

\spacepls

{\large \textbf{g)} Prove that \((A \iff B) \iff B\ \equiv A\):}

\begin{proof}
In the same manner as with the Exclusive OR, logical equivalence is associative. This means that \((A \iff B) \iff B\ \equiv A \iff (B \iff B)\).
Since the RHS of this equivalence is always true by definition (for \(B\) is equivalent to \(B\)), then the value of \(A \iff (B \iff B)\) depends entirely on \(A\).
Therefore, \((A \iff B) \iff B\ \equiv A\).
\end{proof}
\end{document}