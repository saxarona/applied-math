\documentclass[titlepage, letterpaper, fleqn]{article}
\usepackage[utf8]{inputenc}
\usepackage{fancyhdr} % fancy headers, of course!
\usepackage{amsmath} % what do you think?
\usepackage{amsthm} % theorems!
\usepackage{extramarks} % more cute things
\usepackage{enumitem} % i'm not sure...
\usepackage{multicol} % multicolumn...?
\usepackage{amssymb} % more symbols
\usepackage{booktabs} % cool looking tables
\usepackage{tikz} %venn and shizzle
\usepackage{lipsum} %lorem ipsum dolor sit amet f u
\usepackage{mathrsfs} %math script for calligraphic scripting, I GUESS

\topmargin=-0.45in
\evensidemargin=0in
\oddsidemargin=0in
\textwidth=6.5in
\textheight=9.0in
\headsep=0.25in


%
% You should change this things~
%

\newcommand{\mahteacher}{Dr. Viacheslav Kalashnikov}
\newcommand{\mahclass}{Applied Mathematics}
\newcommand{\mahtitle}{Topic II - Activity 6}
\newcommand{\mahdate}{September 14, 2016}
\newcommand{\spacepls}{\vspace{5mm}}
\renewcommand\qedsymbol{\(\blacksquare\)}

%
% Header markings
%

\pagestyle{fancy}
\lhead{1170065 - Xavier Sánchez}
\chead{}
\rhead{}
\lfoot{}
\rfoot{}


\renewcommand\headrulewidth{0.4pt}
\renewcommand\footrulewidth{0.4pt}

\setlength\parindent{0pt}


%
% Create Problem Sections (stolen directly from jdavis/latex-homework-template @ github!)
%

\newcommand{\enterProblemHeader}[1]{
\nobreak\extramarks{}{Problem \arabic{#1} continued on next page\ldots}\nobreak{}
\nobreak\extramarks{Problem \arabic{#1} (continued)}{Problem \arabic{#1} continued on next page\ldots}\nobreak{}
}

\newcommand{\exitProblemHeader}[1]{
\nobreak\extramarks{Problem \arabic{#1} (continued)}{Problem \arabic{#1} continued on next page\ldots}\nobreak{}
\stepcounter{#1}
\nobreak\extramarks{Problem \arabic{#1}}{}\nobreak{}
}

\setcounter{secnumdepth}{0}
\newcounter{partCounter}
\newcounter{homeworkProblemCounter}
\setcounter{homeworkProblemCounter}{1}
\nobreak\extramarks{Exercise \arabic{homeworkProblemCounter}}{}\nobreak{}

% Alias for the Solution section header
\newcommand{\solution}{\textbf{\Large Solution}}

%Alias for the new step section
\newcommand{\steppy}[1]{\textbf{\large #1}}

%
% Homework Problem Environment
%
% This environment takes an optional argument. When given, it will adjust the
% problem counter. This is useful for when the problems given for your
% assignment aren't sequential. See the last 3 problems of this template for an
% example.
%
\newenvironment{homeworkProblem}[1][-1]{
\ifnum#1>0
\setcounter{homeworkProblemCounter}{#1}
\fi
\section{Exercise \arabic{homeworkProblemCounter}}
\setcounter{partCounter}{1}
\enterProblemHeader{homeworkProblemCounter}
}{
\exitProblemHeader{homeworkProblemCounter}
}

%
% Venn diagrams defs
%

% \def\firstcircle{(0,0) circle (1.5cm)}
% \def\secondcircle{(0:2cm) circle (1.5cm)}
% \colorlet{circle edge}{blue!50}
% \colorlet{circle area}{blue!20}

% \tikzset{filled/.style={fill=circle area, draw=circle edge, thick},
%     outline/.style={draw=circle edge, thick}}

%
% My actual info
%

\title{
\vspace{1in}
\textbf{Tecnológico de Monterrey} \\
\vspace{0.5in}
\textmd{\mahclass} \\
\large{\textit{\mahteacher}} \\
\vspace{0.5in}
\textsc{\mahtitle}\\
\textsc{2.1.3 Propositional Logic}\\
\author{01170065  - MIT \\
Xavier Fernando Cuauhtémoc Sánchez Díaz \\
\texttt{mail@gmail.com}}
\date{\mahdate}
}

\begin{document}

\begin{titlepage}
\maketitle
\end{titlepage}

%
% Actual document starts here~
%

\section{Exercise 2.1.3}

{\large \textbf{a)} Prove or disprove \(\models (A \implies B) \vee (B \implies A)\):}

\begin{proof}
Since this formula consists of a disjunction of two subformulas, then if any subformula is true, so is its value.
Now, the first subformula \(A \implies B\) can only be false if \(A\) is true and \(B\) is false.
Let's assume it is false, so we now focus on the second subformula \(B \implies A\).
Then, the subformula \(B \implies A\) will always be true since \(B\) is false, and any implication is true when its hypothesis (\(B\)) is false.

Conversely, assuming that the first subformula \(A \implies B\) is true, then the value of the formula is also true.
Therefore, \((A \implies B) \vee (B \implies A)\) is valid.
\end{proof}

\spacepls

{\large \textbf{b)} Prove or disprove \(\models((A \implies B) \implies B) \implies B\):}

\begin{proof}
This formula is not valid. Consider the following counterexample, when \(v_{\mathscr{I}}(B) = F\) and \(v_{\mathscr{I}}(A) = T\):

\begin{align*}
& ((A \implies B) \implies B) \implies B \equiv \\
& \equiv ((T \implies F) \implies F) \implies F\\
& \equiv (F \implies F) \implies F\\
& \equiv T \implies F\\
& \equiv F
\end{align*}

Therefore, \(\not \models ((A \implies B) \implies B) \implies B\).
\end{proof}

\spacepls

{\large \textbf{c)} Prove or disprove \(\models (A \iff B) \iff (A \iff (B \iff A))\):}

\begin{proof}
This formula is not valid. Consider  the following counterexample with \(v_{\mathscr{I}}(B) = F\). If \(B\) is false, then the first subformula \(A \iff B\) is also false. Thus, the whole formula is false.
Therefore, \(\not \models (A \iff B) \iff (A \iff (B \iff A))\).
\end{proof}

\spacepls

{\large \textbf{d)} Prove or disprove \(\models ((A \wedge B) \implies C) \implies ((A \implies C) \vee (B \implies C))\):}

\begin{proof}
This is a valid formula.
The only way it could yield a false value is if the second subformula\\
\((A \implies C) \vee (B \implies C))\) is false.
This is a disjunction, so both atoms should be false, which is only possible if both \(A\) and \(B\) are assumed to be false and \(C\) is assumed to be true.\\
Assuming the conclusion of the complete formula is false, then we must ensure the hypothesis of the complete formula is true.\\
It follows that \(A \wedge B\) is true, and \(C\) is false, so the whole hypothesis is false. And if the hypothesis is false, then whole formula is true.
Therefore, \(\models ((A \wedge B) \implies C) \implies ((A \implies C) \vee (B \implies C))\).
\end{proof}


\end{document}