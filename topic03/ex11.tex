\documentclass[titlepage, letterpaper, fleqn]{article}
\usepackage[utf8]{inputenc}
\usepackage{fancyhdr} % fancy headers, of course!
\usepackage{amsmath} % what do you think?
\usepackage{amsthm} % theorems!
\usepackage{extramarks} % more cute things
\usepackage{enumitem} % i'm not sure...
\usepackage{multicol} % multicolumn...?
\usepackage{amssymb} % more symbols
\usepackage{MnSymbol} % moar symbols?
\usepackage{booktabs} % cool looking tables
\usepackage{tikz} %venn and shizzle
\usepackage{tikz-qtree-compat} %tableaux
\usepackage{lipsum} %lorem ipsum dolor sit amet f u
\usepackage{mathrsfs} %math script for calligraphic scripting, I GUESS

\topmargin=-0.45in
\evensidemargin=0in
\oddsidemargin=0in
\textwidth=6.5in
\textheight=9.0in
\headsep=0.25in


%
% You should change this things~
%

\newcommand{\mahteacher}{Dr. Viacheslav Kalashnikov}
\newcommand{\mahclass}{Applied Mathematics}
\newcommand{\mahtitle}{Topic III - Activity 11}
\newcommand{\mahdate}{October 26, 2016}
\newcommand{\spacepls}{\vspace{5mm}}
\newcommand{\until}{\mathscr{U}}
\renewcommand\qedsymbol{\(\blacksquare\)}

%
% Header markings
%

\pagestyle{fancy}
\lhead{1170065 - Xavier Sánchez}
\chead{}
\rhead{}
\lfoot{}
\rfoot{}


\renewcommand\headrulewidth{0.4pt}
\renewcommand\footrulewidth{0.4pt}

\setlength\parindent{0pt}


%
% Create Problem Sections (stolen directly from jdavis/latex-homework-template @ github!)
%

\newcommand{\enterProblemHeader}[1]{
\nobreak\extramarks{}{Problem \arabic{#1} continued on next page\ldots}\nobreak{}
\nobreak\extramarks{Problem \arabic{#1} (continued)}{Problem \arabic{#1} continued on next page\ldots}\nobreak{}
}

\newcommand{\exitProblemHeader}[1]{
\nobreak\extramarks{Problem \arabic{#1} (continued)}{Problem \arabic{#1} continued on next page\ldots}\nobreak{}
\stepcounter{#1}
\nobreak\extramarks{Problem \arabic{#1}}{}\nobreak{}
}

\setcounter{secnumdepth}{0}
\newcounter{partCounter}
\newcounter{homeworkProblemCounter}
\setcounter{homeworkProblemCounter}{1}
\nobreak\extramarks{Exercise \arabic{homeworkProblemCounter}}{}\nobreak{}


%Solution Environment

\newenvironment{solution}
{\renewcommand\qedsymbol{$\square$}\begin{proof}[Solution]}
{\end{proof}}

% Alias for the Solution section header
%\newcommand{\solution}{\textbf{\Large Solution}}

%Alias for the new step section
\newcommand{\steppy}[1]{\textbf{\large #1}}

%
% Homework Problem Environment
%
% This environment takes an optional argument. When given, it will adjust the
% problem counter. This is useful for when the problems given for your
% assignment aren't sequential. See the last 3 problems of this template for an
% example.
%
\newenvironment{homeworkProblem}[1][-1]{
\ifnum#1>0
\setcounter{homeworkProblemCounter}{#1}
\fi
\section{Exercise \arabic{homeworkProblemCounter}}
\setcounter{partCounter}{1}
\enterProblemHeader{homeworkProblemCounter}
}{
\exitProblemHeader{homeworkProblemCounter}
}

%
% My actual info
%

\title{
\vspace{1in}
\textbf{Tecnológico de Monterrey} \\
\vspace{0.5in}
\textmd{\mahclass} \\
\large{\textit{\mahteacher}} \\
\vspace{0.5in}
\textsc{\mahtitle}\\
\textsc{3.1.1 Counting and Probability}\\
\textsc{3.1.2 Counting and Probability}\\
\author{01170065  - MIT \\
Xavier Fernando Cuauhtémoc Sánchez Díaz \\
\texttt{mail@gmail.com}}
\date{\mahdate}
}

\begin{document}

\begin{titlepage}
\maketitle
\end{titlepage}

%
% Actual document starts here~
%

\section{Exercise 3.1.1}

{\large \textbf{a)} How many positive two-digit integers are multiples of 3?}

\begin{solution}
The entire set of positive two-digit integers is from 10-99.
There are 7 multiples of 3 between 10 and 30, i.e. 12, 15, 18, 21, 24, 27 and 30.
There are another 10 from 31 to 60, and another 10 from 60 to 90.
Finally, there are three more from 91 to 99.
So, in total, there are 30 positive two-digit integers that are multiples of 3.
\end{solution}

\spacepls

{\large \textbf{b)} What is the probability that a randomly chosen positive two-digit integer is a multiple of 3?}

\begin{solution}
There are 90 positive two-digit integers, from 10 to 99.
30 of those are multiples of 3, and therefore the probability that a randomly chosen positive two-digit integer is a multiple of 3 is of \(\dfrac{30}{90} = \dfrac{1}{3}\).
\end{solution}

{\large \textbf{c)} What is the probability that a randomly chosen positive two-digit integer is a multiple of 4?}

\begin{solution}
There are 8 multiples of 4 between 10 and 40, i.e. 12, 16, 20, 24, 28, 32, 36 and 40.
There are another 10 from 40 to 80.
Finally, there are 4 more from 80 to 99.
Therefore, there are 32 positive two-digit integers that are multiples of 4.
\end{solution}

\spacepls

{\large \textbf{d)} How many positive three-digit integers are multiples of 6?}

\begin{solution}
The entire set of positive three-digit integers comprises all number from 100 to 999.
From 100 to 160, there are 10 multiples of 6.
There are another 10 multiples of 6 from 160 to 220.
This process can be repeated until getting 14 groups of 10, comprising all positive three-digit integers from 100 to 940.
There are 9 more multiples of 6 from 940 to 999.
Therefore, there are 149 positive three-digit integers that are multiples of 6.
\end{solution}

\spacepls

{\large \textbf{e)} What is the probability that a randomly chosen positive three-digit integer is a multiple of 7?}

\begin{solution}
There are 128 positive three-digit integers multiples of 7 (since \(7 \times 128 + 100 = 996\)).
The whole set of positive three-digit integers contains 900 numbers.
Therefore, the probability that a randomly chosen positive three-digit integer is a multiple of 7 is \(\dfrac{128}{900} = \dfrac{32}{225}\).
\end{solution}

\spacepls

\section{Exercise 3.1.2}

{\large Suppose \(A[1], A[2], A[3], \dots, A[n]\) is a one-dimensional array and \(n \geq 50\).\\
\textbf{a)} How many elements are in the array?}

\begin{solution}
Using theorem 9.1.1 from the book of Susanna S.epp, we know that any list of \(n\) numbers has \(n - m + 1\) such that \(m\) is the value of the first element. Hence, if \(n = 50\), then \(|A| = 50 - 1 + 1 = 50\). Therefore, the size of this array will always be \(n\).
\end{solution}

\spacepls

{\large \textbf{b)} How many elements are in the subarray \(A[4], A[5],\dots , A[39]\)?}

\begin{solution}
Using the same theorem 9.1.1, one can easily see that there are \(39 - 4 + 1 = 36\) elements in this subarray.
\end{solution}

\spacepls

{\large \textbf{c)} If \(3 \leq m \leq n\), what is the probability that a randomly chosen array element is in the subarray \(A[3], A[4], \dots, A[m]\)}?

\begin{solution}
Since two subarrays are created (one from 3 to \(m\), and another from \(m+1\) to \(n\)), then the probability of getting an element of the first subarray is the result of \(\dfrac{|A|}{|B|}\) such that \(A,B\) are the first and second subarrays respectively.

\begin{align*}
|A| & = m - 3 + 1 = m - 2 \\
|B| & = n - (m + 1) + 1 = n - m \\[2ex]
P(e) &= \frac{m-2}{n-m}
\end{align*}
\end{solution}

{\large \textbf{d)} What is the probability that a randomly chosen array element is in the subarray shown below if \(n = 39\)?\\
\(A[\frac{n}{2}], A[\frac{n}{2} + 1], \dots , A[n]\)}

\begin{solution}
The total number of elements in the subarray is \(38 - \dfrac{38}{2} + 1 = 20\) since 39 is not even. So the probability of selecting a randomly chosen array element that belongs to this subarray is \(\dfrac{20}{N}\), where \(N\) is the number of the available elements to choose from.
\end{solution}
\end{document}