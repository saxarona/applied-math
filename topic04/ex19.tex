\documentclass[titlepage, letterpaper, fleqn]{article}
\usepackage[utf8]{inputenc}
\usepackage{fancyhdr} % fancy headers, of course!
\usepackage{amsmath} % what do you think?
\usepackage{amsthm} % theorems!
\usepackage{extramarks} % more cute things
\usepackage{enumitem} % i'm not sure...
\usepackage{multicol} % multicolumn...?
\usepackage{amssymb} % more symbols
\usepackage{MnSymbol} % moar symbols?
\usepackage{booktabs} % cool looking tables
\usepackage{tikz} %venn and shizzle
\usepackage{tikz-qtree-compat} %tableaux
\usepackage{lipsum} %lorem ipsum dolor sit amet f u
\usepackage{mathrsfs} %math script for calligraphic scripting, I GUESS

\topmargin=-0.45in
\evensidemargin=0in
\oddsidemargin=0in
\textwidth=6.5in
\textheight=9.0in
\headsep=0.25in


%
% You should change this things~
%

\newcommand{\mahteacher}{Dr. Viacheslav Kalashnikov}
\newcommand{\mahclass}{Applied Mathematics}
\newcommand{\mahtitle}{Topic IV - Activity 19}
\newcommand{\mahdate}{November 16, 2016}
\newcommand{\spacepls}{\vspace{5mm}}
\newcommand{\until}{\mathscr{U}}
\renewcommand\qedsymbol{\(\blacksquare\)}

%
% Header markings
%

\pagestyle{fancy}
\lhead{1170065 - Xavier Sánchez}
\chead{}
\rhead{}
\lfoot{}
\rfoot{}


\renewcommand\headrulewidth{0.4pt}
\renewcommand\footrulewidth{0.4pt}

\setlength\parindent{0pt}


%
% Create Problem Sections (stolen directly from jdavis/latex-homework-template @ github!)
%

\newcommand{\enterProblemHeader}[1]{
\nobreak\extramarks{}{Problem \arabic{#1} continued on next page\ldots}\nobreak{}
\nobreak\extramarks{Problem \arabic{#1} (continued)}{Problem \arabic{#1} continued on next page\ldots}\nobreak{}
}

\newcommand{\exitProblemHeader}[1]{
\nobreak\extramarks{Problem \arabic{#1} (continued)}{Problem \arabic{#1} continued on next page\ldots}\nobreak{}
\stepcounter{#1}
\nobreak\extramarks{Problem \arabic{#1}}{}\nobreak{}
}

\setcounter{secnumdepth}{0}
\newcounter{partCounter}
\newcounter{homeworkProblemCounter}
\setcounter{homeworkProblemCounter}{1}
\nobreak\extramarks{Exercise \arabic{homeworkProblemCounter}}{}\nobreak{}

%Solution Environment
\newenvironment{solution}
{\renewcommand\qedsymbol{$\square$}\begin{proof}[Solution]}
{\end{proof}}

% Alias for the Solution section header
%\newcommand{\solution}{\textbf{\Large Solution}}

%Alias for the new step section
\newcommand{\steppy}[1]{\textbf{\large #1}}

%
% Homework Problem Environment
%
% This environment takes an optional argument. When given, it will adjust the
% problem counter. This is useful for when the problems given for your
% assignment aren't sequential. See the last 3 problems of this template for an
% example.
%
\newenvironment{homeworkProblem}[1][-1]{
\ifnum#1>0
\setcounter{homeworkProblemCounter}{#1}
\fi
\section{Exercise \arabic{homeworkProblemCounter}}
\setcounter{partCounter}{1}
\enterProblemHeader{homeworkProblemCounter}
}{
\exitProblemHeader{homeworkProblemCounter}
}

%
% My actual info
%

\title{
\vspace{1in}
\textbf{Tecnológico de Monterrey} \\
\vspace{0.5in}
\textmd{\mahclass} \\
\large{\textit{\mahteacher}} \\
\vspace{0.5in}
\textsc{\mahtitle}\\
\textsc{Normal, chi-square, F and (Student's) t-distributions}\\
\textsc{4.2.1}\\
\textsc{4.2.2}\\
\textsc{4.2.3}\\
\textsc{4.2.4}\\
\textsc{4.2.5}\\
\textsc{4.2.6}\\
\author{01170065  - MIT \\
Xavier Fernando Cuauhtémoc Sánchez Díaz \\
\texttt{xavier.sanchezdz@gmail.com}}
\date{\mahdate}
}

\begin{document}

\begin{titlepage}
\maketitle
\end{titlepage}

%
% Actual document starts here~
%

\section{Exercise 4.2.1}

{\large If $X$ is a normal random variable with parameters $\mu = 10, \sigma^2 = 36$, compute the following.

\textbf{a)}$P(X > 5)$:}

\begin{solution}
For this exercise, since $\sigma^2 = 36$, then standard deviation is 6.
Then, probability $P(X > 5)= 0.20233$.
\end{solution}

\spacepls

{\large \textbf{b)}$P(4 < X < 16)$:}

\begin{solution}
$P(4 < X < 16)= P(X < 16) - P(X < 4) = 0.84134 - 0.15866 = 0.68268$
\end{solution}

\spacepls

{\large \textbf{b)}$P(X < 8)$:}

\begin{solution}
$P(X < 8) = 0.36944$
\end{solution}

\spacepls

{\large \textbf{b)}$P(X < 20)$:}

\begin{solution}
$P(X < 20) = 0.95221$
\end{solution}

\spacepls

{\large \textbf{b)}$P(X > 16)$:}

$P(X > 16) = 0.15866$

\spacepls

\section{Exercise 4.2.2}

{\large The scholastic Aptitude Test mathematics test scores across the population of high school seniors follow a normal distribution with mean 500 and a standard deviation 100.
If five seniors are randomly chosen, find the probabiilty that all scored below 600.}

\begin{solution}
First we need to input all needed values into the calculator to obtain $P(X < 600) = 0.84134$.
This is the probability that a random senior gets below 600. Now we need to calculate probability of 5 times this independent event: $0.84134^5 \approx 0.42155$
\end{solution}

\spacepls

{\large Exactly three of them scored above 640.}

\begin{solution}
First we need to determine the probability of getting more than 640, which is $P(X>640) = 0.08076$.
Now, getting exactly three means that 2 out of the five will fail to get more than 640,
and this is then multiplied by the number of different combinations of three students out of five chosen,
so it is $\binom{5}{3} \times 0.08076^3 \times 0.91924^2 = 10 \times 0.00044508 = 0.0044508$.
\end{solution}

\spacepls

\section{Exercise 4.2.3}

{\large If $X$ is a chi-square random variable with 6 degrees of freedom, find $P(X \leq 6)$:}

\begin{solution}
$P(X \leq 6) = 0.58$
\end{solution}

\spacepls

{\large Find $P(3 \leq X \leq 9)$:}

\begin{solution}
$P(3 \leq X \leq 9) = P(X < 9) - P(X < 3) = 0.83 - 0.19 = 0.64$
\end{solution}

\spacepls

\section{Exercise 4.2.4}

{\large If $X$ and $Y$ are independent chi-square random variables with 3 and 6 degrees of freedom, respectively, determine the probability that $X + Y$ will exceed 10.}

\begin{solution}
Because both $X$ and $Y$ are independent chi-square random variables, then $X+Y$ has 9 degrees of freedom,
which is the sum of their respective degrees of freedom.
Therefore, the probability that $X+Y$ is greater than 10 is $P(X+Y > 10) =0.35$.
\end{solution}

\spacepls

\section{Exercise 4.2.5}

{\large If $T$ has a t-distribution with 8 degrees of freedom, find $P(T \geq 1)$:}

\begin{solution}
$P(T \geq 1) = 0.1733 $
\end{solution}

\spacepls

{\large Find $P(T \leq 2)$:}

\begin{solution}
$P(T \leq 2) = 0.9597$
\end{solution}

\spacepls

{\large Find $P(-1 < T < 1)$:}

\begin{solution}
$P(-1 < T < 1) = P(T < 1) - P(T < -1) = 0.8267 - 0.1733 = 0.6534$
\end{solution}

\spacepls

\section{Exercise 4.2.6}

{\large If $T_n$ has a t distribution with $n$ degrees of freedom, show that $T_n^2$ has an F-distribution with 1 and $n$ degrees of freedom.}

aa

\end{document}