\documentclass[titlepage, letterpaper]{article} %fleqn if you want left eqs
\usepackage[utf8]{inputenc}
\usepackage{fancyhdr} % fancy headers, of course!
\usepackage{amsmath} % what do you think?
\usepackage{amsthm} % theorems!
\usepackage{extramarks} % more cute things
\usepackage{enumitem} % i'm not sure...
\usepackage{multicol} % multicolumn...?
\usepackage{amssymb} % more symbols
\usepackage{MnSymbol} % moar symbols?
\usepackage{booktabs} % cool looking tables
\usepackage{tikz} %venn and shizzle
\usepackage{tikz-qtree-compat} %tableaux
\usepackage{lipsum} %lorem ipsum dolor sit amet f u
\usepackage{mathrsfs} %math script for calligraphic scripting, I GUESS

\topmargin=-0.45in
\evensidemargin=0in
\oddsidemargin=0in
\textwidth=6.5in
\textheight=9.0in
\headsep=0.25in


%
% You should change this things~
%

\newcommand{\mahteacher}{Dr. Viacheslav Kalashnikov}
\newcommand{\mahclass}{Applied Mathematics}
\newcommand{\mahtitle}{Topic IV - Activity 20}
\newcommand{\mahdate}{November 16, 2016}
\newcommand{\spacepls}{\vspace{5mm}}
\newcommand{\until}{\mathscr{U}}
\renewcommand\qedsymbol{\(\blacksquare\)}

%
% Header markings
%

\pagestyle{fancy}
\lhead{1170065 - Xavier Sánchez}
\chead{}
\rhead{}
\lfoot{}
\rfoot{}


\renewcommand\headrulewidth{0.4pt}
\renewcommand\footrulewidth{0.4pt}

\setlength\parindent{0pt}


%
% Create Problem Sections (stolen directly from jdavis/latex-homework-template @ github!)
%

\newcommand{\enterProblemHeader}[1]{
\nobreak\extramarks{}{Problem \arabic{#1} continued on next page\ldots}\nobreak{}
\nobreak\extramarks{Problem \arabic{#1} (continued)}{Problem \arabic{#1} continued on next page\ldots}\nobreak{}
}

\newcommand{\exitProblemHeader}[1]{
\nobreak\extramarks{Problem \arabic{#1} (continued)}{Problem \arabic{#1} continued on next page\ldots}\nobreak{}
\stepcounter{#1}
\nobreak\extramarks{Problem \arabic{#1}}{}\nobreak{}
}

\setcounter{secnumdepth}{0}
\newcounter{partCounter}
\newcounter{homeworkProblemCounter}
\setcounter{homeworkProblemCounter}{1}
\nobreak\extramarks{Exercise \arabic{homeworkProblemCounter}}{}\nobreak{}

%Solution Environment
\newenvironment{solution}
{\renewcommand\qedsymbol{$\square$}\begin{proof}[Solution]}
{\end{proof}}

% Alias for the Solution section header
%\newcommand{\solution}{\textbf{\Large Solution}}

%Alias for the new step section
\newcommand{\steppy}[1]{\textbf{\large #1}}

%
% Homework Problem Environment
%
% This environment takes an optional argument. When given, it will adjust the
% problem counter. This is useful for when the problems given for your
% assignment aren't sequential. See the last 3 problems of this template for an
% example.
%
\newenvironment{homeworkProblem}[1][-1]{
\ifnum#1>0
\setcounter{homeworkProblemCounter}{#1}
\fi
\section{Exercise \arabic{homeworkProblemCounter}}
\setcounter{partCounter}{1}
\enterProblemHeader{homeworkProblemCounter}
}{
\exitProblemHeader{homeworkProblemCounter}
}

%
% My actual info
%

\title{
\vspace{1in}
\textbf{Tecnológico de Monterrey} \\
\vspace{0.5in}
\textmd{\mahclass} \\
\large{\textit{\mahteacher}} \\
\vspace{0.5in}
\textsc{\mahtitle}\\
\textsc{Parameter Estimation}\\
\textsc{4.3.1}\\
\textsc{4.3.2}\\
\textsc{4.3.3}\\
\textsc{4.3.4}\\
\textsc{4.3.5}\\
\textsc{4.3.6}\\
\author{01170065  - MIT \\
Xavier Fernando Cuauhtémoc Sánchez Díaz \\
\texttt{xavier.sanchezdz@gmail.com}}
\date{\mahdate}
}

\begin{document}

\begin{titlepage}
\maketitle
\end{titlepage}

%
% Actual document starts here~
%

\section{Exercise 4.3.1}

{\large The height of a radio tower is to be measured by measuring both the horizontal distance $X$ from the center of its base to a measuring instrument and the vertical angle of the measuring device.
If five measurements of the distance $X$ give (in feet) the values

150.42, 150.45, 150.49, 150.52, 150.40

and four measurements of the angle $\theta$ give (in degrees) the values

40.26, 40.27, 40.29, 40.26

estimate the height of the tower.}

\begin{solution}
As proved in the book, the maximum estimator of a normal sample is its mean, so we calculate the mean of distances $\overline{X} = 150.456$ and the mean of the angles $\overline{\theta} = 40.27$.

Now, using the tangent of an angle we can estimate the height of the tower as shown:
\[\tan \theta = \frac{\text{opposite}}{\text{adjacent}} \therefore \text{opposite} = \tan\theta \times \text{adjacent}\]

When the values are substituted, we get that the tower is around 127.46059 feet tall.
\end{solution}

\section{Exercise 4.3.2}

{\large An electric scale gives a reading equal to the true weight plus a random error that is normally distributed with mean 0 and standard deviation $\sigma = 0.1$ mg.
Suppose that the results of five successive weightings of the same object are as follows:

3.142, 3.163, 3.155, 3.150, 3.141

\textbf{a)} Determine a 95\% confidence interval estimate of the true weight.}

\begin{solution}
First we need to calculate the mean of the sample: $\overline{x} = 3.1502$.

The 95\% confidence interval estimate is obtained using the following:
\[\overline{x} - 1.96 \frac{\sigma}{\sqrt{n}} < \mu < \overline{x} + 1.96 \frac{\sigma}{\sqrt{n}}\]

The lower bound of the interval is $3.1502 - 1.96\frac{0.1}{\sqrt{5}} \approx 3.06254$,
while the upper bound of the interval is $3.1502 + 1.96\frac{0.1}{\sqrt{5}} \approx 3.23785$.

Because the estimate $\overline{x}$ is within the interval described above,
we can be 95\% certain that the electric scale is reading accurately.
\end{solution}

\spacepls

{\large \textbf{b)} Determine a 99\% confidence interval estimate of the true weight.}

\begin{solution}
The 99\% confidence interval estimate is obtained using the same approach as the above, but using $z_{0.005} = 2.58$.

So, we have $3.1502 - 2.58\frac{0.1}{\sqrt{5}} = 3.03481$ as the lower bound,
and $3.1502 + 2.58\frac{0.1}{\sqrt{5}} = 3.26558$.

Since the estimate $\overline{x} = 3.1502$ is inside the interval described above,
we can be 99\% certain that the electric scale is reading accurately.
\end{solution}

\spacepls

\section{Exercise 4.3.3}

{\large The standard deviation of test scores on a certain achievement test is 11.3.
If a random sample of 81 students had a sample mean score of 74.6,
find a 90\% confidence interval estimate for the average score of all students.}

\begin{solution}
The 90\% confidence interval estimate is obtained using the same approach as above, but using $z_{0.05} = 1.645$.

So we have $74.6 - 1.645\frac{11.3}{\sqrt{81}} = 72.53461$ as the lower bound,
and $74.6 + 1.645\frac{11.3}{\sqrt{81}} = 76.66538$.

Since the mean $\overline{x} = 74.6$ is inside the interval described above,
we can be 90\% certain that the average is a good estimation of the students' scores.
\end{solution}

\spacepls

\section{Exercise 4.3.4}

{\large A sample of 20 cigarettes is tested to determine nicotine content and the average value observed was 1.2 mg.
Compute a 99\% two-sided confidence interval for the mean nicotine content of a cigarette if the sample variance of a cigarette's nicotine content is $s^2=0.04$.}

\begin{solution}
The 99\% confidence interval estimate is obtained using the same approach as above, but using $z_{0.005} = 2.58$.
Before continuing, we need the standard deviation,
which is obtained by calculating the square root of the sample variance:
$s = \sqrt{0.04} = 0.2$

So we have $1.2 - 2.58\frac{0.2}{\sqrt{20}} = 1.08461$ as the lower bound,
and $1.2 + 2.58\frac{0.2}{\sqrt{20}} = 1.31538$ as the upper bound.

Because the average value is 1.2 mg, then we can be 90\% certain that the test is correctly done.
\end{solution}

\spacepls

\section{Exercise 4.3.5}

{\large The capacities (in ampere-hours) of 10 batteries were recorded as follows:

140, 136, 150, 144, 148, 152, 138, 141, 143, 151

\textbf{a)}Estimate the population variance $\sigma^2$.}

\begin{solution}
The population variance can be estimated using $s^2$ as a maximum estimator.
Calculating $s^2$ yields that $\sigma^2 = 32.23333$.
\end{solution}

\spacepls

{\large \textbf{b)} Compute a 99\% two-sided confidence interval for $\sigma^2$.}

Since the population variance $\sigma^2$ is not known,
then to compute a 99\% two-sided confidence interval we need to do the following:
\[\left\{\frac{(n-1)s^2}{\chi_{\alpha /2 ,n-1}^2}, \frac{(n-1)s^2}{\chi_{1 - \alpha /2 , n-1}^2}\right\}\]

Substituting the values, we have $\dfrac{9 \times 32.23333}{23.589} = 12.29810$ as the lower bound,
and $\dfrac{9 \times 32.23333}{1.735} = 167.20459$

We can be 99\% certain that the population variance lies inside that interval.

\spacepls

{\large \textbf{c)} Compute a value $v$ that enables us to state, with 90\% confidence, that $\sigma^2$ is less than $v$.}

\begin{solution}
We need to calculate the lower interval (that is, the upper bound) to make sure that $\sigma^2$ is less than $v$.
For this, we can use the following equation:

\[\frac{(n-1)S^2}{\chi_{1-\alpha,n-1}^2}\]

So we have that the upper bound is $\dfrac{9 \times 32.23333}{4.168} = 69.6017$ and therefore we are 90\% certain that $\sigma^2$ is between 0 and 69.6017.
\end{solution}

\spacepls

\section{Exercise 4.3.6}

{\large Independent random samples are taken from the output of two machines on a production line.
The weight of each item is of interest.
From the first machine, a sample of size 36 is taken, with sample mean of weight of 120 grams and a sample variance of 4.

From the second machine, a sample of size 64 is taken, with a sample mean weight of 130 grams and a sample variance of 5.

It is assumed that the weights of items from the first machine are normally distributed with mean $\mu_1$ and variance $\sigma^2$, and that the weights of items from the second machine are normally distributed with mean $\mu_2$ and variance $\sigma^2$ (that is, the variances are assumed to be equal).

Find a 99\% confidence interval for $\mu_1 - \mu_2$.}

\begin{solution}
The 99\% confidence interval for $\mu_1 - \mu_2$ is calculated as follows:
\[\overline{x} - \overline{y} \pm t_{\alpha /2,n+m-2}{s_p}\sqrt{1/n + 1/m}\]
We need to calculate $s_p$ which is calculated as follows: $s^2_p = \sqrt{\frac{(n-1)s^2_1 + (m-1)s^2_2}{n+m-2}} = \sqrt{\frac{35 \times 4 + 63 \times 5}{98}} = 2.15472$.

So, after substituting the values, we have
\begin{align*}
& 120 - 130 - 2.365 * 2.15472\sqrt{\frac{1}{36}+\frac{1}{64}} \\
& = -11.06164\\[2ex]
& 120 - 130 + 2.365 * 2.15472\sqrt{\frac{1}{36}+\frac{1}{64}} \\
& = -8.93835 \qedhere
\end{align*}
\end{solution}

\end{document}