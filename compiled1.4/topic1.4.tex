\documentclass[titlepage, letterpaper, fleqn]{article}
\usepackage[utf8]{inputenc}
\usepackage{fancyhdr} % fancy headers, of course!
\usepackage{amsmath} % what do you think?
\usepackage{amsthm} % theorems!
\usepackage{extramarks} % more cute things
\usepackage{enumitem} % i'm not sure...
\usepackage{multicol} % multicolumn...?
\usepackage{amssymb} % more symbols
\usepackage{booktabs} % cool looking tables
\usepackage{tikz} %venn and shizzle
\usepackage{lipsum} %lorem ipsum dolor sit amet f u

\topmargin=-0.45in
\evensidemargin=0in
\oddsidemargin=0in
\textwidth=6.5in
\textheight=9.0in
\headsep=0.25in


%
% You should change this things~
%

\newcommand{\mahteacher}{Dr. Viacheslav Kalashnikov}
\newcommand{\mahclass}{Applied Mathematics}
\newcommand{\mahtitle}{Activities 13, 14 \& 15}
\newcommand{\mahdate}{September 1, 2016}
\newcommand{\spacepls}{\vspace{5mm}}
\renewcommand\qedsymbol{\(\blacksquare\)}

%
% Header markings
%

\pagestyle{fancy}
\lhead{1170065 - Xavier Sánchez}
\chead{}
\rhead{}
\lfoot{}
\rfoot{}


\renewcommand\headrulewidth{0.4pt}
\renewcommand\footrulewidth{0.4pt}

\setlength\parindent{0pt}


%
% Create Problem Sections (stolen directly from jdavis/latex-homework-template @ github!)
%

\newcommand{\enterProblemHeader}[1]{
\nobreak\extramarks{}{Problem \arabic{#1} continued on next page\ldots}\nobreak{}
\nobreak\extramarks{Problem \arabic{#1} (continued)}{Problem \arabic{#1} continued on next page\ldots}\nobreak{}
}

\newcommand{\exitProblemHeader}[1]{
\nobreak\extramarks{Problem \arabic{#1} (continued)}{Problem \arabic{#1} continued on next page\ldots}\nobreak{}
\stepcounter{#1}
\nobreak\extramarks{Problem \arabic{#1}}{}\nobreak{}
}

\setcounter{secnumdepth}{0}
\newcounter{partCounter}
\newcounter{homeworkProblemCounter}
\setcounter{homeworkProblemCounter}{1}
\nobreak\extramarks{Exercise \arabic{homeworkProblemCounter}}{}\nobreak{}

% Alias for the Solution section header
\newcommand{\solution}{\textbf{\Large Solution}}

%Alias for the new step section
\newcommand{\steppy}[1]{\textbf{\large #1}}

%
% Homework Problem Environment
%
% This environment takes an optional argument. When given, it will adjust the
% problem counter. This is useful for when the problems given for your
% assignment aren't sequential. See the last 3 problems of this template for an
% example.
%
\newenvironment{homeworkProblem}[1][-1]{
\ifnum#1>0
\setcounter{homeworkProblemCounter}{#1}
\fi
\section{Exercise \arabic{homeworkProblemCounter}}
\setcounter{partCounter}{1}
\enterProblemHeader{homeworkProblemCounter}
}{
\exitProblemHeader{homeworkProblemCounter}
}

%
% Venn diagrams defs
%

% \def\firstcircle{(0,0) circle (1.5cm)}
% \def\secondcircle{(0:2cm) circle (1.5cm)}
% \colorlet{circle edge}{blue!50}
% \colorlet{circle area}{blue!20}

% \tikzset{filled/.style={fill=circle area, draw=circle edge, thick},
%     outline/.style={draw=circle edge, thick}}

%
% My actual info
%

\title{
\vspace{1in}
\textbf{Tecnológico de Monterrey} \\
\vspace{0.5in}
\textmd{\mahclass} \\
\large{\textit{\mahteacher}} \\
\vspace{0.5in}
\textsc{\mahtitle}\\
\textsc{1.4.1: Proof by simple induction}\\
\textsc{1.4.2: Definition by simple recursion}\\
\textsc{1.4.3: Proof by cumulative induction}\\
\author{01170065  - MIT \\
Xavier Fernando Cuauhtémoc Sánchez Díaz \\
\texttt{xavier.sanchezdz@gmail.com}}
\date{\mahdate}
}

\begin{document}

\begin{titlepage}
\maketitle
\end{titlepage}

%
% Actual document starts here~
%

\section{Exercise 1.4.1}

{\large \textbf{a)} Use simple induction to show that for every positive integer \(n\), \(5^n - 1\) is divisible by 4.}

\begin{proof}
\lipsum[1]
\end{proof}

\spacepls

{\large \textbf{b)} Use simple induction to show that for every positive integer \(n\), \(n^3 - n\) is divisible by 3.}
%Hint: in the induction step, you will need to make use of the arithmetic fact that (k + 1)^3 = k^3 +3k^2 + 3k + 1.

\begin{proof}
\lipsum[1]
\end{proof}

\spacepls

{\large \textbf{c)} Show by simple induction that for every natural number \(n\), \(\sum \{2^i \colon 0 \leq i \leq n\} = 2^{n+1} - 1\).}

\begin{proof}
\lipsum[1]
\end{proof}

\section{Exercise 1.4.2}

{\large \textbf{a)} Let \(f \colon \mathbb{N} \to \mathbb{N}\) be the function defined by putting \(f(0) = 0\) and \(f(n+1) = n\) for all \(n \in \mathbb{N}\).}\\
\textbf{i)} Evaluate this function bottom-up for all arguments 0-5.

\begin{proof}
\lipsum[1]
\end{proof}

\spacepls

\textbf{ii)} Explain what \(f\) does by expressing it in explicit terms (i.e. without a recursion).

\spacepls

{\large \textbf{b)} Let \(f \colon \mathbb{N}^+ \to \mathbb{N}\) be  the function that takes each positive integer \(n\) to the greatest natural number \(p\) with \(2^p \leq n\). Define this function by a simple recursion.}
%hint: You will need to divide the recursion step into two cases.

\begin{proof}
\lipsum[1]
\end{proof}

\spacepls


{\large \textbf{c)} Let \(g \colon \mathbb{N} \times \mathbb{N} \to \mathbb{N}\) be defined by putting \(g(m,0) = m\) for all \(m \in \mathbb{N}\) and \(g(m,n+1) = f(g(m,n))\) where \(f\) is defined in the previous part of this exercise.}\\
\textbf{i)} Evaluate \(f(3,4)\) top-down.

\begin{proof}
\lipsum[1]
\end{proof}

\spacepls

\textbf{ii)} Explain what \(f\) does by expressing it in explicit terms.

\begin{proof}
\lipsum[1]
\end{proof}

\section{Exercise 1.4.3}

{\large \textbf{a)} Use cumulative induction to show that any postage cost of four or more pence can be covered by two-pence and five-pence stamps.}

\begin{proof}
\lipsum[1]
\end{proof}

\spacepls

{\large \textbf{b)} Use cumulative induction to show that for every natural number \(n, F(n) \leq 2^n - 1\), where \(F\) is the Fibonacci function.}

\begin{proof}
\lipsum[1]
\end{proof}

\spacepls

{\large \textbf{c)} Calculate \(F(5)\) top-down, and then again bottom-up, where again \(F\) is the Fibonacci function.}

\begin{proof}
\lipsum[1]
\end{proof}

\spacepls

{\large \textbf{d)} Express each of the numbers 14, 15 and 16 as a sum of 3s and/or 8s. Using this fact in your basis, show by cumulative induction that every positive integer \(n \geq 14\) may be expressed as a sum of 3s and/or 8s}.

\begin{proof}
\lipsum[1]
\end{proof}

\spacepls

{\large \textbf{e)} Show by induction that for every natural number \(n, A(1,n) = n+2\), where \(A\) is the Ackermann function.}

\begin{proof}
\lipsum[1]
\end{proof}

\end{document}