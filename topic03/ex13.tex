\documentclass[titlepage, letterpaper, fleqn]{article}
\usepackage[utf8]{inputenc}
\usepackage{fancyhdr} % fancy headers, of course!
\usepackage{amsmath} % what do you think?
\usepackage{amsthm} % theorems!
\usepackage{extramarks} % more cute things
\usepackage{enumitem} % i'm not sure...
\usepackage{multicol} % multicolumn...?
\usepackage{amssymb} % more symbols
\usepackage{MnSymbol} % moar symbols?
\usepackage{booktabs} % cool looking tables
\usepackage{tikz} %venn and shizzle
\usepackage{tikz-qtree-compat} %tableaux
\usepackage{lipsum} %lorem ipsum dolor sit amet f u
\usepackage{mathrsfs} %math script for calligraphic scripting, I GUESS

\topmargin=-0.45in
\evensidemargin=0in
\oddsidemargin=0in
\textwidth=6.5in
\textheight=9.0in
\headsep=0.25in


%
% You should change this things~
%

\newcommand{\mahteacher}{Dr. Viacheslav Kalashnikov}
\newcommand{\mahclass}{Applied Mathematics}
\newcommand{\mahtitle}{Topic III - Activity 13}
\newcommand{\mahdate}{October 26, 2016}
\newcommand{\spacepls}{\vspace{5mm}}
\newcommand{\until}{\mathscr{U}}
\renewcommand\qedsymbol{\(\blacksquare\)}

%
% Header markings
%

\pagestyle{fancy}
\lhead{1170065 - Xavier Sánchez}
\chead{}
\rhead{}
\lfoot{}
\rfoot{}


\renewcommand\headrulewidth{0.4pt}
\renewcommand\footrulewidth{0.4pt}

\setlength\parindent{0pt}


%
% Create Problem Sections (stolen directly from jdavis/latex-homework-template @ github!)
%

\newcommand{\enterProblemHeader}[1]{
\nobreak\extramarks{}{Problem \arabic{#1} continued on next page\ldots}\nobreak{}
\nobreak\extramarks{Problem \arabic{#1} (continued)}{Problem \arabic{#1} continued on next page\ldots}\nobreak{}
}

\newcommand{\exitProblemHeader}[1]{
\nobreak\extramarks{Problem \arabic{#1} (continued)}{Problem \arabic{#1} continued on next page\ldots}\nobreak{}
\stepcounter{#1}
\nobreak\extramarks{Problem \arabic{#1}}{}\nobreak{}
}

\setcounter{secnumdepth}{0}
\newcounter{partCounter}
\newcounter{homeworkProblemCounter}
\setcounter{homeworkProblemCounter}{1}
\nobreak\extramarks{Exercise \arabic{homeworkProblemCounter}}{}\nobreak{}

%Solution Environment
\newenvironment{solution}
{\renewcommand\qedsymbol{$\square$}\begin{proof}[Solution]}
{\end{proof}}

% Alias for the Solution section header
%\newcommand{\solution}{\textbf{\Large Solution}}

%Alias for the new step section
\newcommand{\steppy}[1]{\textbf{\large #1}}

%
% Homework Problem Environment
%
% This environment takes an optional argument. When given, it will adjust the
% problem counter. This is useful for when the problems given for your
% assignment aren't sequential. See the last 3 problems of this template for an
% example.
%
\newenvironment{homeworkProblem}[1][-1]{
\ifnum#1>0
\setcounter{homeworkProblemCounter}{#1}
\fi
\section{Exercise \arabic{homeworkProblemCounter}}
\setcounter{partCounter}{1}
\enterProblemHeader{homeworkProblemCounter}
}{
\exitProblemHeader{homeworkProblemCounter}
}

%
% My actual info
%

\title{
\vspace{1in}
\textbf{Tecnológico de Monterrey} \\
\vspace{0.5in}
\textmd{\mahclass} \\
\large{\textit{\mahteacher}} \\
\vspace{0.5in}
\textsc{\mahtitle}\\
\textsc{3.3.1 Axioms of Probability}\\
\textsc{3.3.2 Axioms of Probability}\\
\textsc{3.3.3 Expected Value}\\
\textsc{3.3.4 Expected Value}\\
\author{01170065  - MIT \\
Xavier Fernando Cuauhtémoc Sánchez Díaz \\
\texttt{xavier.sanchezdz@gmail.com}}
\date{\mahdate}
}

\begin{document}

\begin{titlepage}
\maketitle
\end{titlepage}

%
% Actual document starts here~
%

\section{Exercise 3.3.1}

{\large Suppose \(A\) and \(B\) are mutually exclusive events in a sample space \(S\). \(C\) is another event in \(S, A \cup B \cup C = S\), and \(A\) and \(B\) have probabilities 0.4 and 0.2, respectively.\\
\textbf{a)} What is \(P(A \cup B)\)?}

\begin{solution}
Since \(A,B\) are mutually exclusive events, then the probability of both events happening (\(P(AB)\)) is \(P(A) \times P(B) = 0.08\).
We also know that \(P(A) \cup P(B)\) is \(P(A) + P(B) - P(AB) = 0.4 + 0.2 - 0.08 = 0.52\).
\end{solution}

\spacepls

{\large \textbf{b)} Is it possible that \(P(C) = 0.2\)? Explain.}

\begin{solution}
This can't be possible, since \(S = A \cup B \cup C\), and because \(A \cup B = 0.52\), then \(C = 1 - 0.52 = 0.48\). According to the axioms of probability, the sum of all events should equal 1.
\end{solution}

\spacepls

\section{Exercise 3.3.2}

{\large Suppose \(A\) and \(B\) are events in a sample space \(S\) with probabilities 0.8 and 0.7, respectively. Suppose also that \(P(A \cap B) = 0.6\). What is \(P(A \cup B)\)?}

\begin{solution}
\(P(A \cup B) = P(A) + P(B) - P(AB)\) so  \(P(A \cup B) = 0.8 + 0.7 - 0.6 = 0.9\).
\end{solution}

\spacepls

\section{Exercise 3.3.3}

{\large A company sends millions of people an entry form for a sweepstakes accompanied by an order form for a magazine subscription.
The first, second and third prizes are \$ 10,000,000, \$ 1,000,000 and \$ 50,000 respectively.
In order to qualify for a prize, a person is not required to order any magazines but has to spend 60 cents to mail back the entry form.
If 30 millions people qualify by sending back their entry forms, what is a persons expected gain or loss?}

\begin{solution}
The expected gain of this lottery is the sum of all probabilities multiplied by its value –the prize minus the price.
This is \(\dfrac{1}{30,000,000} (1 \cdot \$ 9,999,999.4 + 1 \cdot \$ 999,999.4 + 1 \cdot \$ 49,999.4 + 29,999,997 \cdot - \$ 0.6)\) which equals \(-0.231666\dots\).
On average, a person will lose around 23 cents per try.
\end{solution}

\spacepls

\section{Exercise 3.3.4}

{\large An urn contains four balls numbered 2, 2, 5 and 6.
If a person selects a set of two balls at random, what is the expected value of the sum of the numbers on the balls?}

\begin{solution}
There are \(\binom{4}{2} = 6\) different ways to get two balls from the urn: \{(2a,2b), (2a,5), (2a,6), (2b,5), (2b,6), (5,6)\}.
Therefore, the value of the sum of each pair is \{4, 7, 8, 7, 8, 11\}.
This means that the expected value is \(\dfrac{1}{6} \cdot 4 + \dfrac{2}{6} \cdot 7 + \dfrac{2}{6} \cdot 8 + \dfrac{1}{6} \cdot 11 = 7.5\).
\end{solution}
\end{document}