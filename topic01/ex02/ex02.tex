\documentclass[titlepage, letterpaper, fleqn]{article}
\usepackage[utf8]{inputenc}
\usepackage{fancyhdr} % fancy headers, of course!
\usepackage{amsmath} % what do you think?
\usepackage{amsthm} % theorems!
\usepackage{extramarks} % more cute things
\usepackage{enumitem} % i'm not sure...
\usepackage{multicol} % multicolumn...?
\usepackage{amssymb} % more symbols
\usepackage{booktabs} % cool looking tables
\usepackage{tikz} %venn and shizzle

\topmargin=-0.45in
\evensidemargin=0in
\oddsidemargin=0in
\textwidth=6.5in
\textheight=9.0in
\headsep=0.25in


%
% You should change this things~
%

\newcommand{\mahteacher}{Dr. Viacheslav Kalashnikov}
\newcommand{\mahclass}{Applied Mathematics}
\newcommand{\mahtitle}{Activity 4 - Exercise 1.2.1: Cartesian Products}
\newcommand{\mahdate}{August 31, 2016}
\newcommand{\spacepls}{\vspace{5mm}}
\renewcommand\qedsymbol{\(\blacksquare\)}

%
% Header markings
%

\pagestyle{fancy}
\lhead{1170065 - Xavier Sánchez}
\chead{}
\rhead{}
\lfoot{}
\rfoot{}


\renewcommand\headrulewidth{0.4pt}
\renewcommand\footrulewidth{0.4pt}

\setlength\parindent{0pt}


%
% Create Problem Sections (stolen directly from jdavis/latex-homework-template @ github!)
%

\newcommand{\enterProblemHeader}[1]{
\nobreak\extramarks{}{Problem \arabic{#1} continued on next page\ldots}\nobreak{}
\nobreak\extramarks{Problem \arabic{#1} (continued)}{Problem \arabic{#1} continued on next page\ldots}\nobreak{}
}

\newcommand{\exitProblemHeader}[1]{
\nobreak\extramarks{Problem \arabic{#1} (continued)}{Problem \arabic{#1} continued on next page\ldots}\nobreak{}
\stepcounter{#1}
\nobreak\extramarks{Problem \arabic{#1}}{}\nobreak{}
}

\setcounter{secnumdepth}{0}
\newcounter{partCounter}
\newcounter{homeworkProblemCounter}
\setcounter{homeworkProblemCounter}{1}
\nobreak\extramarks{Exercise \arabic{homeworkProblemCounter}}{}\nobreak{}

% Alias for the Solution section header
\newcommand{\solution}{\textbf{\Large Solution}}

%Alias for the new step section
\newcommand{\steppy}[1]{\textbf{\large #1}}

%
% Homework Problem Environment
%
% This environment takes an optional argument. When given, it will adjust the
% problem counter. This is useful for when the problems given for your
% assignment aren't sequential. See the last 3 problems of this template for an
% example.
%
\newenvironment{homeworkProblem}[1][-1]{
\ifnum#1>0
\setcounter{homeworkProblemCounter}{#1}
\fi
\section{Exercise \arabic{homeworkProblemCounter}}
\setcounter{partCounter}{1}
\enterProblemHeader{homeworkProblemCounter}
}{
\exitProblemHeader{homeworkProblemCounter}
}

%
% Venn diagrams defs
%

% \def\firstcircle{(0,0) circle (1.5cm)}
% \def\secondcircle{(0:2cm) circle (1.5cm)}
% \colorlet{circle edge}{blue!50}
% \colorlet{circle area}{blue!20}

% \tikzset{filled/.style={fill=circle area, draw=circle edge, thick},
%     outline/.style={draw=circle edge, thick}}

%
% My actual info
%

\title{
\vspace{1in}
\textbf{Tecnológico de Monterrey} \\
\vspace{0.5in}
\textmd{\mahclass} \\
\large{\textit{\mahteacher}} \\
\vspace{0.5in}
\textsc{\mahtitle}
\author{01170065  - MIT \\
Xavier Fernando Cuauhtémoc Sánchez Díaz \\
\texttt{mail@gmail.com}}
\date{\mahdate}
}

\begin{document}

\begin{titlepage}
\maketitle
\end{titlepage}

%
% Actual document starts here~
%

{\large \textbf{a)} Show that \(A \times (B \cap C) = (A \times B) \cap (A \times C)\).}

\begin{proof}
	Let \((a,b) \in A \times (B \cap C)\).\\
	So, by definition, \(a \in A\) and \(b \in B\) and also \(b \in C\).\\
	That means that \((a,b) \in A \times B\) and \((a,b) \in A \times C\).\\
	Therefore, \((a,b) \in (A \times B) \cap (A \times C)\).\\
	\\
	Now assume \((a,b) \in (A \times B) \cap (A \times C)\).\\
	That means \((a,b) \in A \times B\) and also \((a,b) \in A \times C\).\\
	Because \((a,b) \in A \times B\), then \(a \in A\) and also \(b \in B\).\\
	And because \((a,b) \in A \times C\), then \(b \in B\).\\
	So if \(b \in B\) and \(b \in C\), then \(b \in B \cap C\),\\
	and so \((a,b) \in A \times (B \cap C)\).\\
	Therefore, \(A \times (B \cap C) = (A \times B) \cap (A \times C)\).
\end{proof}

\spacepls

{\large \textbf{b)} Show that \(A \times (B \cup C) = (A \times B) \cup (A \times C)\).}

\begin{proof}
	Let \((a,b) \in A \times (B \cup C)\).\\
	So, by definition, \(a \in A\) and \(b \in B \cup C\).\\
	If \(b \in B\), then \((a,b) \in A \times B\).\\
	If \(b \in C\), then \((a,b) \in A \times C\).\\
	So, \((a,b) \in (A \times B) \cup (A \times C)\).\\
	\\
	Now assume \((a,b) \in (A \times B) \cup (A \times C)\).\\
	If \((a,b) \in A \times B\), then \(a \in A\) and \(b \in B\).\\
	If \((a,b) \in A \times C\), then \(b \in C\).\\
	And if \(b \in B\) or \(b \in C\), then \(b \in B \cup C\).\\
	And, by definition, \((a,b) \in A \times (B \cup C)\).\\
	Therefore, \(A \times (B \cup C) = (A \times B) \cup (A \times C)\).
\end{proof}

\end{document}